\section{Dílčí kroky procesu směrování uvnitř směrovače.}

\newpage
\section{Směrovací protokol Distance-Vector, směrovací protokol RIP}

\newpage
\section{Směrovací protokoly Link-State, směrovací protokol OSPF.}

\newpage
\section{Základní funkční bloky mechanismů pro zajištění kvality služeb.}

\newpage
\section{Způsob využití základních typů rámců technologie WiFi. Účel a dílčí kroky jednotlivých fází připojení klienta do sítě WLAN (skenování, autentizace, asociace, stav připojení, odpojení, roaming).}

\newpage
\section{Deterministické a náhodné přístupové metody využívané v sítích WLAN.}

\newpage
\section{Vlastnosti základních algoritmů výběru buněk (PIM, iRRM, SLIP, DRRM).}

\newpage
\section{Vlastnosti a struktura přepínače se sdíleným mediem a se sdílenou pamětí.}

\newpage
\section{Struktura spojovacích polí s prostorovým dělením kanálu (křížový přepínač, plně propojená struktura, Banyan).}

\newpage
\section{Výhody a nevýhody architektur pro přepojovací uzly využívající: vstupní vyrovnávací paměti, výstupní vyrovnávací paměti, sdílenou paměť, a virtuální výstupní fronty.}