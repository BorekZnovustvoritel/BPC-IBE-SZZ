\section{Veřejnoprávní ochrana autorského práva}

Ochrana práva se dělí na soukromoprávní a~veřejnoprávní. Nejčastěji je narušení autorského práva řešeno v~občanskoprávním sporu. Spadá pod ochranu duševního vlastnictví.

% TODO Když jsou zásahy do autorského práva řešeny především soukromoprávně, v jakých případech je to veřejnoprávně?

Veřejnoprávní rovina se dělí na~správněprávní, trestněprávní a~ústavněprávní.

Odpovědnost se dělí na:

\begin{itemize}
    \item Občanskoprávní, která spadá do soukromoprávní roviny.
    \item Přestupková, spadá do veřejnoprávní roviny (správněprávní).
    \item Trestní, spadá do veřejnoprávní roviny (trestněprávní).
\end{itemize}

Výše škody:

\begin{itemize}
    \item Nikoliv nepatrná škoda dosahující částky 10\,000~Kč.
    \item Nikoliv malou škodu dosahující částky 50\,000~Kč.
    \item Škodou dosahující částky 100\,000~Kč.
    \item Značnou škodou dosahující částky 1\,000\,000~Kč.
    \item Škodou velkého rozsahu dosahující částky 10\,000\,000~Kč.
\end{itemize}

Od~minimální výše škody se odvíjí sazba v~případě trestního řízení a~specifikace zda se jedná pouze o~přestupek nebo trestný čin.

Do~\textbf{správněprávní} spadají přestupky, řeší je obce s~rozšířenou působností. Přestupku se dle
\href{https://www.zakonyprolidi.cz/cs/2000-121#p105a-1-a}{§105 AZ} dopustí FO, PO nebo podnikající FO, která neoprávněně užije autorské dílo, umělecký výkon, zvukový či zvukově obrazový záznam, rozhlasové nebo televizní vysílání nebo databázi. U~FO lze uložit sankci do~100\,000~Kč, nebo 150\,000~Kč dle spáchaného přestupku. U~PO a~podnikajících FO lze uložit sankci do~50\,000~Kč, 100\,000~Kč, 150\,000~Kč, nebo 500\,000~Kč dle spáchaného přestupku. Přestupky na~rozdíl od~trestných činů mohou trestat větší okruh jednání (nedbalost; neúmyslnost při~povinnosti vědět). Trestný čin může být spáchán pouze a~jedině úmyslně.

\textbf{Trestní právo} z~pohledu porušení autorského práva je popsáno v~\href{https://www.zakonyprolidi.cz/cs/2009-40#p270}{§270~TZ}. Při~porušení autorského práva není potřeba způsobit minimální škodu, na~rozdíl od~skutkových podstat, ale často je k ní přihlíženo v případě určování trestu. Účelem trestněprávního práva je potrestat osobu, která se dopustila porušení práv a~žádat náhradu způsobené škody.

Sazba za~porušení může být dle \href{https://www.zakonyprolidi.cz/cs/2009-40#p270}{§270~TZ} odnětí svobody až na~2 roky, zákaz činnosti nebo propadnutí věci. Dále, pokud by porušení autorského zákona vykazovalo znaky obchodní činnosti nebo jiného podnikaní (je-li porušení ve~značném rozsahu), mění se sazba na~6~měsíců až 5~let odnětí svobody, peněžitý trest nebo propadnutí věci. Pokud by pachatel porušil autorská práva ve~velkém rozsahu, může být potrestán na~3--8 roků odnětí svobody.

Osoba, které bylo narušeno autorské právo, podává trestní oznámení na~policii, poté už \uv{nic neřeší}. Probíhá \textbf{přípravné řízení}, ve~kterém policie šetří co se stalo a~shromažďují se důkazy. V~této fázi se také určí jestli došlo vůbec k~trestnému činu. Pokud se jedná trestný čin, podává se \textbf{obžaloba}. Po~podání obžaloby je vyrozuměn obviněný, obhájce a~poškozený. Potom probíhá soudní řízení, kdy je rozhodnuto o~nevině nebo vině obžalovaného. V~případě autorské práva se v~průběhu musí vyřešit, do~jakého autorského práva bylo zasaženo a~k~uznání viny musí být odůvodněno přesným paragrafem (blanketní skutková podstata).

\subsection{Celní správa}

Celní správa se stará o~boj proti padělání a~pirátství. Celní orgány jsou oprávněny zasáhnout proti zboží a~umožnit držitelům práv duševního vlastnictví jejich ochranu. Zboží porušující právo duševního vlastnictví bude zničeno, pokud nebylo bezúplatně převedeno k~humanitárním účelům nebo pokud není použito v~rámci působnosti orgánů celní správy.

\textbf{Padělek} je zboží včetně jeho obalu na němž jsou bez souhlasu majitele ochranné známky umístěno označení stejné nebo zaměnitelné s~originální ochranou známkou a~veškeré věci nesoucí stejné nebo obdobné označení.

\textbf{Nedovolená napodobenina} je zboží včetně obalu, které bylo přímo nebo nepřímo bez souhlasu majitele nebo spolumajitele patentu, majitele autorského práva a~další, které je kopií nebo v~němž je kopie zahrnuta. Do toho spadá i~forma nebo raznice určená nebo upravená k~výrobě nedovolené napodobeniny.

Zjednodušeně padělek se snaží vydávat za originál zatímco nedovolená napodobenina je kopií originálu. Kopie filmu, nebo softwaru by měla být označena jako nedovolená napodobenina.






\clearpage
\section[Elektronický podpis a~elektronická pečeť -- právní úprava a~druhy]{Elektronický podpis a~elektronická pečeť -- \newline právní úprava a druhy}

\subsection{eIDAS}

Cílem \href{https://eur-lex.europa.eu/legal-content/CS/TXT/?uri=CELEX\%3A32014R0910}{nařízení Evropského parlamentu a~Rady č.\,910/2014 eIDAS} (\emph{Electronic Identification and Services}) je vytvoření jednotného rámce pro elektronickou identifikaci jednotlivých subjektů. Dalším důležitým cílem je zvýšení důvěrnosti elektronických transakcí na vnitřním trhu. V~České republice toto nařízení je adaptováno v podobě \href{https://www.zakonyprolidi.cz/cs/2016-297}{zákona č.\,297/2016}.

\textbf{Identifikace}: postup používaní osobních údajů, které jedinečně identifikují určitou osobu. \\
\textbf{Autentizace}: postup, který umožňuje potvrdit identifikaci nebo původ a~integritu dat v~elektronické podobě. \\
\textbf{Projev vůle}.

\subsection{Elektronický podpis}

Elektronický podpis dle eIDAS jsou data v~elektronické podobě, která jsou připojena k~jiným datům v~elektronické podobě nebo jsou s~nimi logicky spojena a~která podepisující osoba používá k~podepsání.

Elektronickému podpisu nesmějí být upírány právní účinky a~nesmí být odmítán v~soudním sporu z~důvodu že má elektronickou podobu nebo nesplňuje požadavky na~kvalifikovaný elektronický podpis.

Elektronické podpisy se dělí na:

\begin{itemize}
    \item Elektronický podpis (prostý)
    \item Zaručený elektronický podpis
    \item Uznávaný elektronický podpis
    \item Kvalifikovaný elektronický podpis
\end{itemize}

(Prostý) \textbf{elektronický podpis} je základní a~nejjednodušší typ elektronického podpisu. Řadí se sem napsání jména a~příjmení na~konec elektronického dokumentu nebo emailu, vzor podpisu vytvořený elektronickou tužkou%
\footnote{Třeba podpis na~tlakovou destičku, např. v~bankách.} %
a~další. Nejčastěji se využije mezi dvěma soukromoprávními subjekty, kdy nahrazuje obyčejný podpis.

\textbf{Zaručený elektronický podpis} má dle nařízení eIDAS splňovat následující požadavky:

\begin{itemize}
    \item Je jednoznačně spojen s podepisující osobou.
    \item Umožňuje identifikaci podepisující osoby.
    \item Je vytvořen pomocí dat pro vytváření elektronických podpisů, která podepisující osoba může s~vysokou úrovní důvěry použít pod svou výhradní kontrolou.
    \item Je k~datům, která jsou tímto podpisem podepsaná, připojen takovým způsobem, že je možné zjistit jakoukoliv následnou změnu dat.
\end{itemize}

Nevýhodou tohoto podpisu, i~když se zdá být bezpečný, je že podepisující osoba může vytvořit podpis jakékoliv osoby. Tento typ podpisu nezaručuje, kdo je autorem podpisu.

\textbf{Uznávaný elektronický podpis} není zmíněn v~nařízení eIDAS, ale vychází z~české adaptace nařízení. Jedná se o~elektronický podpis založený na kvalifikovaném cerfitikátu pro elektronický podpis. Na rozdíl od zaručeného elektronického podpisu musí být potvrzen certifikační autoritou. Toto zaručuje spojitost s~osobou již byl vystaven. Pro komunikaci s~veřejnoprávním subjektem elektronicky je nutné použít tento nebo vyšší typ elektronického podpisu. Pokud má osoba zřízenou datovou schránku tak, elektronického podpisu nemusí využívat.

\textbf{Kvalifikovaný elektronický podpis} je dle nařízení eIDAS nejvyšší formou elektronického podpisu. V~nařízení je definován jako zaručený elektronický podpis, který je vytvořen kvalifikovaným prostředkem pro vytváření elektronických podpisů a~který je založen na kvalifikovaném certifikátu pro elektronické podpisy. Na rozdíl od uznávaného elektronického je při vytváření nutné použít kvalifikovaný prostředek pro vytváření elektronických podpisů, který je na samostatném nosiči a~je nepřenositelný na jiný nosič (USB token, čipová karta, eID). Takto nelze odcizit identitu na~dálku, musí být odcizen nosič samotný.

\subsubsection{Poskytovatelé služeb vytvářející důvěru}

\textbf{Poskytovatel služeb vytvářejících důvěru} je FO nebo PO, která poskytuje jednu či více služeb vytvářejících důvěru buď jako kvalifikovaný, nebo jako nekvalifikovaný poskytovatel služeb vytvářejících důvěru.

\textbf{Kvalifikovaný poskytovatel služeb vytvářejících důvěru} je poskytovatel služeb vytvářejících důvěru, který poskytuje jednu či více kvalifikovaných služeb vytvářejících důvěru a~kterému orgán dohledu udělil status kvalifikovaného poskytovatele.

V České republice vydávají kvalifikované elektronické podpisy společnosti:
\begin{itemize}
    \item První certifikační autorita,
    \item Česká pošta,
    \item eIdentity.
\end{itemize}

\subsection{Elektronická pečeť}

Data v elektronické podobě připojená k~jiným (pečetěným) datům v~elektronické podobě s~cílem zaručit jejich pravost a~neměnnost.

Existují 4 druhy elektronických pečetí:
\begin{itemize}
    \item Prostá elektronická pečeť.
    \item Zaručená elektronická pečeť.
    \item Zaručená elektronická pečeť s certifikátem.
    \item Kvalifikovaná elektronická pečeť.
\end{itemize}

Z~těchto typů pečetí je v~EU povinně uznávána pouze kvalifikovaná elektronická pečeť a~je vytvářena podobně jako kvalifikovaný elektronický podpis. Elektronická pečeť by měla sloužit k~potvrzení vydání elektronického dokumentu určitou právnickou osobou. Pečetě nejsou spojené s~určitou fyzickou osobou na rozdíl od~elektronického podpisu (rozdíl PO a~FO). Elektronickou pečeť lze krom podepisovaní dokumentů také využít k~podpisu elektronického časového razítka.

\subsection{Elektronické časové razítko}

Data v~elektronické podobě, která spojují jiná data v~elektronické podobě s~určitým okamžikem a~prokazují, že tato jiná data existovala v~daném okamžiku.







\clearpage
\section{Datové schránky -- právní úprava a~praxe používání}

Datové schránky upravuje \href{https://www.zakonyprolidi.cz/cs/2008-300}{zákon č.\,300/2008 o~elektronických úkonech a autorizované konverzi dokumentů}. Dle tohoto zákona je datová schránka definována jako elektronické úložiště, které je určeno ke komunikaci s~orgány veřejné moci mezi sebou nebo s~FO, podnikajícími FO nebo PO. FO a~PO mohou komunikovat i~mezi sebou. Datové schránky zřizuje ministerstvo vnitra, provozovatelem je Česká pošta.

\textbf{Datové zprávy} jsou dokumenty orgánů veřejné moci doručované prostřednictvím datové schránky, úkony prováděné vůči orgánům veřejné moci prostřednictvím datové schránky a~dokumenty FO, podnikajících FO a~PO dodávané prostřednictvím datové schránky. Pokud je dokument nebo úkon určen do~vlastních rukou, odesílatel to musí vyznačit v~datové zprávě.

Datová schránka je v~nařízení eIDAS definována jako služba elektronického doporučeného doručovaní. Tato služba je v~nařízení popsána jako služba umožnující přenášet data mezi třetími osobami elektronickými prostředky a~poskytuje důkazy týkající se nakládání s~přenášenými daty, včetně dokladu o~odeslání a~přijetí dat a~která chrání přenášená data před rizikem ztráty, odcizení, poškození nebo neoprávněné změně.

Ze~zákona je datová schránka zřizována:

\begin{itemize}
    \item orgánům veřejné moci,
    \item PO zřízeným zákonem,
    \item PO zapsaným v~obchodním rejstříku,
    \item podnikajícím FO:
          \begin{itemize}
              \item advokát,
              \item daňový poradce,
              \item insolvenční správce,
              \item statutární auditor,
              \item znalec,
              \item soudní překladatel nebo tlumočník.
          \end{itemize}
\end{itemize}

Zbylé FO, podnikající FO a~PO nemají si povinnost zřídit datovou schránku, ale můžou o~její vytvoření zažádat. Žádost o~vytvoření by měla být vyřízena do~3~pracovních dnů ode dne podání žádosti. Žádost se podává online přes portál eIDENTITA.CZ nebo osobně na CzechPointu nebo písemně v~listiné podobě na~Ministerstvo vnitra.

% kvalifikovaný prostředek
Proces vytvoření a~zpřístupnění začíná podáním žádosti. Po~vyřízení žádosti jsou přístupové údaje zaslány do~vlastních rukou, případně pomocí kvalifikovaného prostředku (FO). Po~převzetí údajů je datová schránka zpřístupněna prvním přihlášením nebo 15.~dnem od~obdržení údajů.

Při~ztrátě nebo odcizení údajů je povinnost poskytovatele neprodleně zneplatnit přístupové údaje a~zaslat nové.

Datová schránka je zrušena po uplynutí 3~let od~úmrtí FO, podnikající FO, PO nebo od~zrušení orgánu veřejné moci.

\subsection{Komunikace}

\subsubsection{Orgán veřejné moci}

Je-li to možné, orgány veřejné moci spolu komunikují prostřednictvím datové schránky, pokud se nedoručuje na~místě. V~případě že má FO, podnikající FO, nebo PO zřízenou datovou schránku, orgán veřejné moci má povinnost pomoci ní komunikovat, pokud se nedoručuje na místě nebo veřejnou vyhláškou.

\textbf{Okamžikem doručení} se počítá doba přihlášení osoby. Pokud se osoba do~datové schránky nepřihlásí do~10 dni od~dodání, za~doručený je dokument považován posledním dnem této lhůty (neplatí, vylučuje-li to jiný právní předpis). Doručení má stejné právní účinky jako doručení do~vlastních rukou.

\subsubsection{Nepovinné osoby}

Osoby bez~povinnosti mít vytvořenou datovou schránku (ale vytvořenou ji mají) nejsou povinni pomocí ní komunikovat s~orgány veřejné moci. Okamžikem doručení se počítá stejně jako v~případě orgánů veřejné moci okamžik přihlášení osoby. Doba bez~přihlášení kdy se považuje zpráva za~doručenou už v~zákoně není uvedena.

\subsubsection{Formáty a další náležitosti datové schránky}

Pro~komunikaci s~přidáním přílohy k~datové zprávě je omezený počet typů příloh (podporovány jsou základní typy dokumentů jako pdf, doc/docx, xls/xlsx, xml a~další). Maximální velikost datové zprávy je 20~MB. V~datové schránce je zpráva uložena po~dobu 90 dnů od~doby přihlášení nebo, v~případě že nedojde k~přihlášení, po~dobu 3~let. Pokud datová zpráva není v~přípustném formátu, převyšuje maximální velikost nebo obsahuje škodlivý kód, správce tuto zprávu nepřijme k~odeslaní.













\clearpage
\section{Provozní a~lokalizační údaje a~jejich využití v trestním řízení}

Při~vyšetřování trestné činnosti se používá pojem \textbf{data retention}, který znamená shromažďování provozních a~lokalizačních údajů ve formě metadat. Cílem data retention je zajistit zpětnou dostupnost v~případě potřeby. Zpráva přenášená přes síť obsahuje metadata a~obsah. Operátoři mohou sbírat pouze metadata a~zpracovávat je mohou pouze za~stanovených podmínek. Samotný obsah zpracováván být nesmí a~musí být zaručeno, že nebude zpracováván ani nikým jiným.

Provozní a~lokalizační údaje mohou vypovídat o~soukromém životě jedince. Tyto údaje mohou být také automaticky zpracovávány, na~rozdíl od~odposlechů. Lze z~nich vytvářet komunikační profily jedince, lze díky nim předvídat i~chovaní či~obsah komunikace. Proto byl sběr těchto dat Evropským soudem pro~lidská označen označen jako narušení ochrany soukromí. Do~zásahu do~tohoto práva musí být sledován legitimní cíl a~opatření k~němu musí být přiměřená.

V~České republice upravuje provozní a~lokalizační údaje \href{https://www.zakonyprolidi.cz/cs/2005-127}{Zákon č.\,127/2005 o~elektronických komunikacích}.

\textbf{Provozní údaje} jsou jakékoliv údaje zpracovávané pro potřeby přenosu zprávy sítí elektronických komunikací nebo pro její účtování \href{https://www.zakonyprolidi.cz/cs/2005-127#p90-1}{§90~ZoEK}.

\textbf{Lokalizační údaje} jsou jakékoliv údaje zpracovávané v~síti elektronických komunikací nebo službou elektronických komunikací, které určují zeměpisnou polohu telekomunikačního koncového zařízení uživatele veřejně dostupné služby elektronických komunikací \href{https://www.zakonyprolidi.cz/cs/2005-127#p91-1}{§91~ZoEK}.

V~\href{https://www.zakonyprolidi.cz/cs/2005-127#p97-3}{§97 odst.\,3 ZoEK} je stanoveno, že provozovatel komunikační sítě nebo služby elektronických komunikací je povinen po~dobu šesti měsíců uchovávat provozní a~lokalizační údaje které jsou vytvořeny nebo zpracovávány. Po~uplynutí této doby musí být data smazána. Poskytovatelé mají za~toto uchovávaní nárok na náhradu nákladů, které se řeší ve~\href{https://www.zakonyprolidi.cz/cs/2013-462/zneni-20140101}{vyhlášce č.\,462/2013}.

Povinnost poskytovat tyto data orgánům činným v~trestním řízení je definována v~\href{https://www.zakonyprolidi.cz/cs/1961-141#p88a}{§88a zákona č.\,141/1961} (Trestní řád). Uschovávání, předávání a~likvidaci provozních a~lokalizačních údajů upravuje \href{https://www.zakonyprolidi.cz/cs/2012-357}{vyhláška č.\,357/2012}.

Údaje o~komunikaci si mohou vyžádat orgány činné v~trestním řízení, Policie ČR, Bezpečnostní informační služba, Vojenské zpravodajství a~Česká národní banka.

Nejčastěji je k~získání těchto údajů potřeba žádost podepsaná soudcem, ale existují i~výjimky. Každá z~těchto organizací si může vyžádat údaje při~splnění podmínek zvláštního právního předpisu.

Právo uchovávat provozní a~lokalizační údaje mají pouze subjekty poskytující komunikační sítě a~služby elektronických komunikací. Tyto údaje musí uchovávat po~dobu šesti měsíců a~poté je zlikvidovat. Výjimku z~likvidace tvoří data o~která požádal některý z~orgánů uvedených výše. Pokud byly tyto údaje získány špatným postupem nebo od~subjektu který nemá právo tyto údaje uchovávat, v~soudním řízení se považují za~neplatné. Nejčastěji to bývá IP adresa.


















\clearpage
\section[Právní úprava kybernetické bezpečnosti v ČR -- základní principy]{Právní úprava kybernetické bezpečnosti v~ČR --\newline základní principy}

% TODO Zmínit krizový zákon?
% https://www.zakonyprolidi.cz/cs/2000-240

Kybernetická bezpečnost je definována \href{https://www.zakonyprolidi.cz/cs/2014-181}{zákonem č. 181/2014 o kybernetické bezpečnosti} (ZoKB); prvky kritické infrastruktury jsou definovány v~\href{https://www.zakonyprolidi.cz/cs/2010-432}{nařízení vlády č. 432/2010}.

Kybernetická bezpečnost je aktivita, která se zabývá ochranou počítačových systémů před poškozením a narušením provozu hardwaru, softwaru nebo informací (souhrn prostředků směřujících k~zajištění ochrany kybernetického prostoru). Jejím primárním cílem je zajistit CIA triádu (\emph{confidentiality}\,--\,důvěrnost, \emph{integrity}\,--\,integrita, \emph{availability}\,--\,dostupnost) spravovaných sítí. Hlavním smyslem je pak ochrana prostředí k~realizaci informačních práv člověka.

\textbf{Kybernetický prostor} je digitální prostředí umožnující vznik, zpracovaní a výměnu informací tvořené informačními systémy, službami a sítěmi elektronických komunikací. Je definovaný v~\href{https://www.zakonyprolidi.cz/cs/2014-181#p2-1-a}{§2 odst.\,1 písm.\,a) ZoKB}.

Povinnost zajišťovat kybernetickou bezpečnost mají:
\begin{itemize}
    \item Soukromé subjekty:
          \begin{itemize}
              \item poskytovatelé služeb elektronických komunikací \href{https://www.zakonyprolidi.cz/cs/2014-181#p3-1-a}{§3 odst.\,1 písm.\,a) ZoKB},
              \item osoby zajišťující významnou síť \href{https://www.zakonyprolidi.cz/cs/2014-181#p3-1-b}{§3 odst.\,1 písm.\,b) ZoKB}.
          \end{itemize}
    \item Soukromé a veřejné subjekty:
          \begin{itemize}
              \item správce informačních systému/komunikačních systémů kritické informační infrastruktury \href{https://www.zakonyprolidi.cz/cs/2014-181#p3-1-c}{§3 odst.\,1 písm.\,c) a písm.\,d) ZoKB}.
          \end{itemize}
    \item Veřejné subjekty:
          \begin{itemize}
              \item správce významného informačního systému \href{https://www.zakonyprolidi.cz/cs/2014-181#p3-1-e}{§3 odst.\,1 písm.\,e) ZoKB}.
          \end{itemize}
    \item Síťová informační služba (NIS):
          \begin{itemize}
              \item poskytovatel základních služeb \href{https://www.zakonyprolidi.cz/cs/2014-181#p3-1-f}{§3 odst.\,1 písm.\,f) ZoKB},
              \item poskytovatel digitální služby \href{https://www.zakonyprolidi.cz/cs/2014-181#p3-1-h}{§3 odst.\,1 písm.\,h) ZoKB}.
          \end{itemize}
\end{itemize}

\textbf{Kritickou infrastrukturou} se rozumí prvek nebo systém prvků kritické infrastruktury, jehož narušení by mělo závažný na~bezpečnost a~chod státu. Můžou do~toho spadat některá z~odvětví jako energetika, doprava, zdravotnictví, digitální infrastruktura a~další spadající do~kategorií definovaných v~\href{https://www.zakonyprolidi.cz/cs/2014-181\#p2-1-i}{§2 odst.\,1 písm.\,i) ZoKB}.

\textbf{Kritickou informační infrastrukturou} jsou kybernetické složky kritické infrastruktury. Například ovládací systémy (SCADA) nebo obecně cyber-physical systemy (CPS).

\textbf{Prvkem kritické infrastruktury} je například elektrárna nebo v~případě poskytovatelů elektronických komunikačních služeb jejich síť.

\textbf{Provozovatelem kritické služby} je vlastník daného prvku kritické infrastruktury.

Provozovatel má určité povinnosti, které se odvíjí od~toho, co přesně provozuje. Těmito povinnostmi jsou:

\begin{itemize}
    \item Obecné povinnosti (sdělení kontaktních údajů, mít vytvořená organizační a~technická bezpečnostní opatření) se vztahují na~většinu provozovatelů a~jsou preventivního charakteru. Je to nejnižší stupeň přijímaných opatření.
    \item Operativní povinnosti (hlášení incidentů, varování, reaktivní a~ochranná protiopatření) slouží pro~případ spolupráce při~stavu ohrožení/incidentu. Zde už musí být nastaveny mechanizmy pro~detekci incidentů.
    \item Koordinace při~stavu kybernetického nebezpečí. Při~vyhlášení se rozšiřují povinnosti ve~vztahu k~vládnímu a~národnímu CERTu.
    \item Požadavky na dodavatele.
    \item Certifikace.
\end{itemize}

NÚKIB může vydat opatření:

\begin{itemize}
    \item Varovaní \href{https://www.zakonyprolidi.cz/cs/2014-181#p12}{§12 ZoKB}. Je-li nalezena hrozba/zranitelnost/riziko v oblasti kyberbezpečnosti.
    \item Reaktivní opatření \href{https://www.zakonyprolidi.cz/cs/2014-181#p13}{§13 ZoKB}. Jestliže se něco děje tak vydává opatření aby povinné subjekty reagovaly určitým způsobem.
    \item Ochranné opatření \href{https://www.zakonyprolidi.cz/cs/2014-181#p14}{§14 ZoKB}. Vydává se opatření až po proběhlém bezpečnostním incidentu, aby se povinné subjekty zabezpečily.
\end{itemize}

Protože ne všechny subjekty mají povinnost provádět reaktivní a~ochranná opatření, existuje \textbf{stav kybernetického nebezpečí}. Při~rozsáhlejším útoku potřebuje stát reagovat efektivně a~takto ho rozšíří na~všechny ostatní subjekty, na~které vyšší typy opatření normálně neplatí. Tento stav vyhlašuje ředitel NÚKUBu s~maximální délkou na~sedm dní s~prodloužením až na~30 dní.

Bezpečnost pro~organizace zajišťují CSIRT týmy, které zabezpečují svoji izolovanou infrastrukturu. Na~úrovní státu bezpečnost zajišťují národní a~vládní CERT týmy, které koordinují postupy provozovatelů sítí a~systémů. Na~mezinárodní úrovni jsou to různé mezinárodní organizace a~dobrovolné spolky, které především umožňují sdílení informací (například ENISA).

\textbf{Vládní CERT} je definován v~\href{https://www.zakonyprolidi.cz/cs/2014-181#p20}{§20 ZoKB}. Spadá pod~něj NCKB (Národní centrum kybernetické bezpečnosti)/GovCERT.CZ, který je částí NÚKIBU. Přijímá prohlášení o~kybernetických a~bezpečnostních incidentech, vyhodnocuje údaje o~kybernetických bezpečnostních událostech, poskytuje součinnost při~výskytu incidentu/události, přijímá údaje od~provozovatele národního CERTu a~vyhodnocuje je. Také spolupracuje s~CSIRT týmy jiných států.

\textbf{Národní CERT} je definován v~\href{https://www.zakonyprolidi.cz/cs/2014-181#p17}{§17 ZoKB} a~v~ČR ho provozuje CZ.NIC na~základně veřejnoprávní smlouvy. Přijímá oznámení kontaktních údajů a hlášení o kybernetických bezpečnostních incidentech. Vyhodnocuje kybernetické bezpečnostní incidenty u povinných subjektů. Působí jako kontaktní místo a podobně jako vládní spolupracuje s CSIRT týmy.







\clearpage
\section{Ochrana osobních údajů -- vymezení osob a~základní principy zpracování osobních údajů}

\href{https://eur-lex.europa.eu/legal-content/CS/TXT/HTML/?uri=CELEX:32016R0679}{Nařízení 2016/679 o~ochraně fyzických osob v~souvislosti se~zpracováním osobních údajů [\dots] (GDPR)} a~česká implementace
\href{https://www.zakonyprolidi.cz/cs/2019-110}{Zákon č.~110/2019 Sb., o~zpracování osobních údajů}.

GDPR vychází z~\href{https://eur-lex.europa.eu/legal-content/cs/TXT/HTML/?uri=CELEX:31995L0046}{Data Protection Directive} z~roku 1995.

Hlavními prvky jsou prevenční princip (široká aplikace úpravy a~úzká aplikace výjimek, limitace a~vázanost účelem zpracování, \enquote{co není dovoleno je zakázáno}) a~obecnost (dopadá na~všechny právní osoby od~živnostníků přes~nadnárodní společnosti po~státy). Dodržování se nekontroluje aktivně, ale v~případě vyšetřování musí být veškerá dokumentace v~pořádku.

\subsection*{Pojmy}

\begin{itemize}
    \item \textbf{Osobní údaje}: veškeré informace o identikované/identifikovatelné fyzické osobě, kterou lze přímo či~nepřímo identifikovat.
    \item \textbf{Subjekt údajů}: fyzická osoba.
    \item \textbf{Zpracování}: sběr, uchovávání, distribuce a~veškerá manipulace s~údaji.
    \item \textbf{Správce}: určovatel zpracování.
    \item \textbf{Zpracovatel}: vykonavatel zpracování (může jít o~stejnou osobu).
    \item \textbf{Účel} a~\textbf{Prostředky} zpracování.
\end{itemize}

\subsection*{Důvody zpracování}

\begin{itemize}
    \item \textbf{Souhlas subjektu údajů}.
    \item \textbf{Smluvní závazek}: nutnost k~plnění smlouvy.
    \item \textbf{Právní povinnost}: vyplývající ze~zákona, např. držení kontaktních údajů při~obchodním styku.
    \item \textbf{Ochrana životně důležitých zájmů subjektu nebo FO}.
    \item \textbf{Veřejný zájem nebo výkon veřejné moci}.
    \item \textbf{Oprávněný zájem} správce nebo třetí osoby: např. ochrana infrastruktury.
\end{itemize}

\subsection{Principy GDPR}

\textbf{Vztahuje se na~hodně kategorií dat.} Směřuje k~ochraně všech osobních údajů: \emph{\enquote{Identifikovatelnou FO lze přímo či~nepřímo identifikovat jménem, číslem, lokačními údaji, síťovým identifikátorem nebo jedním či~více prvky fyzické, fyziologické, genetické, psychické, ekonomické, kulturní nebo společenské identity.}}. Omezená \enquote{implementace}: předpokládá se doplnění místní státní legislativou. Osobním údajem je i~IP/MAC adresa, UID nebo jiný (pseudo)unikátní identifikátor, pomocí jeho samotného nebo spojením s~ostatními může dojít k~identifikaci.

\textbf{Extrateritorialita.} Platí pro~FO/PO se sídlem v~EU, osoby nabízející služby EU rezidentům nebo monitorující jejich chování. Vztahuje se tedy i na~subjekty mimo EU. Subjekty ze~zahraničí musí mít v~některém ze~států EU svého zástupce.

\textbf{Vztahuje se i~na~zpracovatele}. Jsou povinni informovat o~uschování dat (jaká, za~jakým účelem, kde jsou, jak jsou zabezpečená) a~ohlašovat porušení zabezpečení. Data mohou zpracovávat pouze na~pokyn správce, je vynucena kontrola bezpečnosti subdodavatelů, mlčenlivost, nápomoc při~výkonu práv subjektu či vymazání dat.

\textbf{Důraz na~odpovědnost správců}. \emph{Privacy by design}. \emph{Privacy by default}. Správci se nemusí registrovat, mohou uschovávat pouze minimum údajů vyplývající z~důvodů zpracování. Musí dodržovat bezpečnost opatření: pseudonymizace a~šifrování, CIA\footnote{Confidentiality, Integrity, Availability} triáda, zálohování, revize ochrany.

\textbf{Posílení práv subjektu údajů.} Práva na~přístup, opravu, smazání a~blokování. GDPR posiluje nutnost jednoznačného souhlasu, zahrnuje právo na~zapomnění či~ochranu proti profilování. \\
Souhlas musí být samostatný a~srozumitelný, prokazuje jej správce; pro~subjekty mladší 16 (13) let existují zvláštní pravidla. Informace o~zpracování musí obsahovat účely, kategorie údajů, příjemce, dobu uschování, musí informovat o~právu na~výmaz, o~existenci profilování nebo automatického rozhodování.

\textbf{Oznamování narušení bezpečnosti.} Musí proběhnout do~72 hodin. Informování subjektu pokud existuje vysoké riziko, úřadu při~jakémkoliv narušení: povaha, kategorie, počet subjektů a~údajů, důsledky, přijatá opatření.

\textbf{Předávání dat mimo EU.} Bylo velmi omezeno. Koncept odpovídající ochrany, vhodných záruk a~povolení.

\textbf{Pokuty}. Až~€20M nebo 4\,\% celosvětového obratu (dle toho co je vyšší). Auditování dozorovými úřady. \emph{One Stop Shop} (možnost kontaktovat místní úřad pro~ochranu osobních údajů, i~když je správce ve~státě jiném).

\clearpage
\section{Povinnosti správce a~zpracovatele osobních údajů}
\label{question-7}

Správce chce sbírat osobní údaje, a~to za~předem definovaným účelem. Více správců může spravovat jedna data; každý z~nich zodpovídá sám za~sebe.

\emph{\enquote{Správce zavede vhodná technická a~organizační opatření, aby zajistil a~byl schopen doložit, že je zpracování prováděno v~souladu s~nařízením GDPR. Tato opatření musí být podle potřeby revidována a aktualizována.}}

\emph{\enquote{S~přihlédnutím ke~stavu techniky, nákladům na~provedení, povaze, rozsahu, kontextu a~účelům zpracování [\dots] zavede správce [\dots] vhodná technická a organizační opatření, jako je pseudonymizace, jejichž účelem je provádět zásady ochrany údajů, jako je minimalizace údajů [\dots].}}

Musí být zpracovávány pouze osobní údaje nezbytně nutné pro~konkrétní účel zpracování. Tyto údaje nesmí být standardně bez~zásahu člověka zpřístupněny neomezenému počtu fyzických osob.

\vspace*{1em}

V~okamžiku získání osobních údajů správce poskytne%
\footnote{%
    dle \href{https://eur-lex.europa.eu/legal-content/CS/TXT/HTML/?uri=CELEX:32016R0679\#d1e2243-1-1}{2016/679/ES, čl. 13}  (GDPR)%
}:

\begin{itemize}
    \item totožnost a~kontaktní údaje správce (a~jeho zástupce), případně kontaktní údaje pověřence pro~ochranu osobních údajů
    \item účely zpracování a~jejich právní základ
    \item příjemce nebo kategorie příjemců osobních údajů
    \item úmysl předat osobní údaje do~třetí země nebo mezinárodní organizaci
    \item dobu, po~kterou budou osobní údaje uloženy, případně kritéria pro~stanovení takové doby
    \item existenci práva požadovat přístup k~osobním údajům týkajícím se subjektu údajů
    \item existenci práva odvolat souhlas, podat stížnost u~dozorového úřadu
    \item jestli jde o~zákonný nebo smluvní požadavek, zda má subjekt možnost údaje neposkytnout a~důsledky neposkytnutí
\end{itemize}

Základní povinnosti správce osobních údajů vyplývající z GDPR jsou:

\begin{itemize}
    \item Odpovědnost
          \begin{itemize}
              \item Za~dodržování zásad zpracování
              \item Za~dodržování povinností upravených nařízením
              \item Za~zabezpečení údajů
          \end{itemize}
    \item Povinnosti
          \begin{itemize}
              \item Aplikace standardní ochrany osobních údajů
              \item Jmenovat pověřence pro ochranu osobních údajů
              \item Posuzovat vliv~na~ochranu osobních údajů
              \item Ohlásit případy porušení zabezpečení osobních údajů příslušnému úřadu a~postiženým osobám
              \item Vést záznamy
          \end{itemize}
\end{itemize}

Zpracovatel je osoba/firma, která je najata správcem osobních údajů. Zpracovatel nemusí vždy existovat nebo jich může být více. Například je možné, že se osobní údaje nezpracovávají, nebo si je správce zpracovává sám. Pokud zpracovatel začne rozhodovat o~účelu dat sám, stává se správcem. Mezi těmito entitami musí při~zpracovaní být vždy sepsána písemná smlouva ve~které je stanoven předmět a~doba trvání zpracování, povaha a~účel zpracování, typ osobních údajů a~kategorie subjektů údajů, povinnosti a~práva správce.

Povinnosti zpracovatele:

\begin{itemize}
    \item Zpracovávat osobní údaje pouze na základě pokynů správce.
    \item Zajišťuje mlčenlivost osob zpracovávající údaje.
    \item Zabezpečuje je stejně jako správce
    \item Při rozhodnutí správce buď údaje vymaže nebo je vrátí správci po ukončení služeb spojených se zpracováním.
\end{itemize}

\emph{Firma A, která prodává zboží a~sbírá osobní údaje zákazníků, např. věk, a~tím se stává správcem osobních údajů. Firma A chce zpracovat analýzu věku zákazníků nejčastěji nakupujících určité položky. Firma A zadá zpracování firmě B, aby jí data zpracovala. Firma B se tím pádem stává zpracovatelem těchto osobních údajů.}

Ke~zpracování osobních údajů je nutno uvést souhlas, kde by mělo být uvedeno, co je shromažďováno a~za jakým účelem. Při~potřebě může osoba, o~které jsou shromažďovány osobní údaje, požádat o~vymazání z~databáze nebo jen přístup k datům o~ní vedeným a správce je musí smazat nebo poskytnout. Jsou i~výjimky, kdy lze zpracovávat osobní údaje bez~souhlasu, musí k~nim ale existovat zákonný důvod.

Příklad výjimek:

\begin{itemize}
    \item plnění smlouvy,
    \item plnění právní povinnosti -- uchování faktury,
    \item výkon veřejné moci,
    \item ochrana~životně důležitých zájmů subjektu údajů nebo jiné FO -- lékař uschovává informace o~léčbě,
    \item plnění úkolu prováděného ve~veřejném zájmu,
    \item plnění nezbytné pro~účely oprávněných zájmů příslušného správce -- například půjčování peněz.
\end{itemize}





\clearpage
\section{Práva subjektu údajů ve~vztahu ke~správci a~zpracovateli}

\href{https://eur-lex.europa.eu/legal-content/CS/TXT/HTML/?uri=CELEX:32016R0679#d1e2150-1-1}{Nařízení 2016/679 (GDPR), kapitola III: Práva subjektu údajů}.

Správce poskytuje údaje stručně, transparentně, srozumitelně a~snadno přístupným způsobem. Informace poskytne bez zbytečného odkladu a~nejdéle do~jednoho měsíce od~obdržení žádosti (v~případě potřeby, složitosti a~vysokém počtu žádostí lze prodloužit o~dva měsíce). Veškeré úkony jsou bezplatné; pokud jsou žádosti nedůvodné nebo nepřiměřené (opakují se), může správce uložit poplatek zohledňující náklady, nebo může žádosti odmítnout vyhovět.

\subsection*{Informování}

Dle~článků 13 a~14 má subjekt právo být informován o~sběru osobních údajů (viz otázku~\ref{question-7}), dle~článku 15 má právo na~přístup ke~sbíraným údajům.

Dle~článku 16 má právo na~opravu nepřesných a na~doplnění neúplných osobních údajů.

\subsection*{Výmaz a~omezení}

Dle článku 17 má právo na~výmaz (\enquote{právo být zapomenut}), pokud jeho údaje již nejsou potřebné pro~deklarované účely, subjekt odvolává souhlas, vznese námitku proti~zpracování nebo jsou jeho údaje zpracovávány protiprávně. Toto právo se neuplatní, pokud je zpracování nezbytené pro~výkon práva na~svobodu projevu a~informace, pro~splnění právní povinnosti nebo pro~splnění úkolu provedeného ve~veřejném zájmu nebo při~výkonu veřejné moci nebo pro~určení, výkon nebo obhajobu právních nároků.

Dle článku 18 má právo na~omezení zpracování, pokud subjekt popírá přesnost osobních údajů (to na~dobu potřebnou k~ověření přesnosti správcem), zpracování je~protiprávní a~subjekt odmítá výmaz osobních údajů, správce údaje nepotřebuje, ale subjekt je požaduje pro výkon právních nároků.

\subsection*{Marketing, profilování a~automatické zpracování}

Dle článku 21 má právo vznést námitku proti~zpracování včetně profilování. Subjekt má vždy právo vznést námitku proti~zpracování pro~účely marketingu.

Subjekt má právo nebýt předmětem rozhodnutí založeného výhradně na~automatizovaném zpracování. Toto právo se nepoužije pokud to není nezbytné k~uzavření nebo plnění smlouvy, pokud je to povoleno právem Unie/členského státu nebo pokud je založeno na~výslovném souhlasu subjektu.

\clearpage
\section{Informace veřejného sektoru -- pojem, česká a~evropská právní úprava}

S~pojmem informace veřejného sektoru se lze setkat v~souvislosti s~právem na informace, které bylo v~průběhu let definováno několika úmluvami a~doporučeními. Všechny se shodují v~tom, že každá osoba má právo na~informace veřejného sektoru, na~práci s~nimi, na~jejich kontrolu nebo zkoumání.

V~Evropské unii je právo na~přístup k~informacím zahrnuto v~listině základních práv a~svobod Evropské unie, kde je to definováno jako \enquote{právo na přístup k~dokumentům orgánů, institucí a~jiných subjektů unie, a~to bez ohledu na~to, na~jakém nosiči se dokumenty nacházejí.}

Dále právo na informace umožňuje směrnice evropského parlamentu a~rady \href{https://eur-lex.europa.eu/legal-content/CS/TXT/?uri=celex\%3A32003L0098}{2003/98/ES o~opakovaném použití informací veřejného sektoru (\uv{PSI směrnice})}, která má novelizovanou podobu ve~formě \href{https://eur-lex.europa.eu/legal-content/EN/TXT/?uri=CELEX:02003L0098-20130717}{2013/37/EU}. Novelizovaná směrnice nezakládá povinnost poskytovat informace (dokumenty), ale zaměřuje se na~jejich použitelnost. Míří k~podpoře otevřených dat.

Dokumentem se rozumí obsah (i~jakákoliv část) na~jakémkoliv nosiči (psaný či tištěný na~papíře či uložený v~elektronické formě nebo jako zvuková, vizuální nebo audiovizuální nahrávka). Směrnice se nevztahuje na:

\begin{itemize}
    \item dokumenty, které nepřísluší do~oblasti veřejných úkolů daného subjektu,
    \item dokumenty zatížené právy duševního vlastnictví třetích stran,
    \item dokumenty s~omezeným přístupem z~důvodů ochrany osobních údajů,
    \item dokumenty, které mají v držení kulturní instituce jiné než knihovny, muzea a archivy.
\end{itemize}

V~roce 2019 byla schválena \href{https://eur-lex.europa.eu/legal-content/CS/TXT/?uri=CELEX:32019L1024}{Směrnice 2019/2014, o~otevřených datech a~opakovaném použití informací veřejného sektoru}, která má přepracované znění (recast) směrnice 2003/98/ES. Třemi hlavními novinkami jsou:

\begin{itemize}
    \item rozšíření povinných subjektů o~veřejné podniky,
    \item rozšíření působnosti na~výzkumná data,
    \item povinné poskytování informací s vysokou hodnotou.
\end{itemize}

Rozšíření povinných subjektů o~veřejné podniky vychází ze~směrnice \href{https://eur-lex.europa.eu/legal-content/en/TXT/?uri=CELEX:32014L0025}{2014/25/EU}. Veřejný podnik je definován jako podnik ve~kterém veřejní zadavatelé mohou vykonávat přímo nebo nepřímo dominantní vliv na~základě vlastnických práv k~podniku, finanční účasti v~něm nebo pravidel jimiž se řídí. O~dominantní vliv veřejných zadavatelů se jedná pokud v~kterémkoliv z~níže uvedených případů tito veřejní zadavatelé přímo či nepřímo: a)~drží většinu upsaného základního kapitálu podniku; b)~disponují většinou hlasovacích práv vyplývajících z~podílu na~podniku; c)~mohou jmenovat více než polovinu členů správního, řídícího nebo dozorčího orgánu podniku.

V~České republice je úprava o~poskytnutí informací definována v~listině základních práv a~svobod. Toto právo může být omezeno pokud je to nezbytné pro~ochranu práv a~svobod druhých, bezpečnosti státu, veřejnou bezpečnost, ochranu veřejného zdraví a~mravnosti.

Dále je povinnost poskytovat informace uvedena v~\href{https://www.zakonyprolidi.cz/cs/1999-106}{zákonu č.\,106/1999} a~neaplikuje se, pokud existuje zvláštní předpis (zákon o~právu na~informace o~životním prostředí, živnostenský zákon atd.). Tento zákon upravuje, kdo má poskytovat informace, jak si o~ně zažádat a~jakou formou mají být poskytnuta. Dle zákona 106/1999 jsou povinné subjekty:

\begin{itemize}
    \item Státní orgány, územní samosprávné celky a~jejich orgány a veřejné instituce.
    \item Subjekty, kterým zákon svěřil pravomoc rozhodovat o~právech a~povinnostech osob, v~rozsahu výkonu této pravomoci.
\end{itemize}

Poskytovat informace lze buď na~žádost, nebo zveřejněním, kdy se zveřejnění dělí na~povinné a~dobrovolné. Do~povinného spadají všechny povinné subjekty, které musí zveřejňovat alespoň minimum informací dané zákonem 106/1999 (co všechno se má zveřejňovat je uvedeno v~§5). Některé výjimky z~aplikace 106/1999:

\begin{itemize}
    \item utajování informací,
    \item soukromí a osobní udaje (platy zaměstnanců),
    \item obchodní tajemství,
    \item autorské právo třetích osob,
    \item a~další.
\end{itemize}

\clearpage
\section{Otevřená data -- pojem a~právní úprava}

Jde o~úplná, snadno dostupná, strojově čitelná data používající standardy s~volně dostupnou specifikací, která jsou zpřístupněna za~jasně daných podmínek užití s~minimem omezení, dostupná uživatelům při~vynaložení minima úsilí.

Zákon č.~106/1999 Sb., o~svobodném přístupu k~informacím, §3 (11): \emph{\enquote{Otevřenými daty se rozumí informace zveřejňované v~otevřeném a~strojově čitelném formátu, jejichž způsob ani účel následného využití není omezen a~jsou evidovány v~národním katalogu otevřených dat.}}

Stupeň otevřenosti dat k určení stupně se používají hvězdy 1-5, kdy 1 je nehorší:

\begin{itemize}
    \item data jsou online (nejsou strojově čitelná [PDF])
    \item data jsou online ve~strukturované podobě (ve strojové podobě [excel])
    \item data jsou online ve~strukturované podobě v~otevřeném formátu (data jsou zpřístupněna v~neproprietárním otevřeném formátu [CSV, JSON])
    \item data jsou online ve~strukturované podobě v~otevřeném formátu a~mají vlastní IRI\footnote{Internationalized Resource Identifier}
    \item data jsou online ve~strukturované podobě v~otevřeném formátu, mají vlastní IRI a~jsou přímo propojená s~dalšími datovými zdroji (linkování) (data jsou nalinkována na jiná data, čím se poskytne kontext)
\end{itemize}

\href{https://www.zakonyprolidi.cz/cs/1999-106#p3-7}{§3 odst.~7}: strojově čitelným formátem se rozumí formát datového souboru s~takovou strukturou, která umožňuje programovému vybavení snadno nalézt, rozpoznat a~získat z~tohoto datového souboru konkrétní informace, včetně jednotlivých údajů a~jejich vnitřní struktury.

\href{https://www.zakonyprolidi.cz/cs/1999-106#p3-8}{§3 odst.~8}: otevřeným formátem se rozumí formát datového souboru, který není závislý na~konkrétním technickém a~programovatelném vybavení a~je zpřístupněn veřejnosti bez~jakéhokoli omezení, které by znemožňovalo využití informací obsažených v~datovém souboru.

Povinná data jsou například jízdní řády, metadata registru smluv nebo příjemci dotací.

Mezi právními problémy jsou např. špatně nastavené smlouvy s~provozovateli dat (vendor lock-in -- \href{http://ictjudikatura.law.muni.cz/wiki/6_As_38/2015_-_51_-_\%C5\%BD\%C3\%A1dost_o_specifick\%C3\%BD_form\%C3\%A1t_informac\%C3\%AD_(CHAPS)}{spor CHAPS v.~Seznam o~jízdní řády}), smlouvy zastaralé nebo neexistující, práva duševního vlastnictví (autorské právo, právo pořizovatele databáze -- u~státních dat neexistují a~nic licencovat není třeba), anonymizace (dle GDPR).

Např.~\href{https://mapaexekuci.cz}{mapa exekucí}, \href{https://prazdnedomy.cz}{prázdné domy}, \href{https://hlidacstatu.cz}{Hlídač státu}, \href{https://data.brno.cz}{data.Brno}.
