\section{Prvočísla -- popište způsob generování a uveďte příklad pravděpodobnostního testu a skutečného testu}

\newpage
\section{Teorie čísel, algebraické struktury -- popište účel a~způsob výpočtu Eulerovy funkce. Popište požadavky na grupu a způsob generování grupy pro algoritmy založené na problému DL.}

\newpage
\section{Modulární aritmetika -- popište algoritmus Square and Multiply a Čínskou větu o zbytcích.}

\newpage
\section{Symetrická kryptografie -- proudové šifry, synchronní a asynchronní proudové šifry.}
\label{question-4}
% Promised by jedla97

\newpage
\section{Blokové šifry -- Product Ciphers, konstrukce, Feistelova síť, DES, AES, základní módy blokových šifer.}
% Promised by jedla97

\newpage
\section{Asymetrické algoritmy -- RSA, Diffie-Hellman, ECDH systém a jejich využití pro digitální podpis.}

Problém diskrétního logaritmu%
\footnote{\emph{Discrete Logarithm Problem}, DLP}%
: $$c = m^n \Mod p$$ je pro velká čísla neřešitelné.

\subsection[RSA]{\href{https://en.wikipedia.org/wiki/RSA_(cryptosystem)}{RSA}}

Jde o~relativně pomalý algoritmus, proto se s~ním nešifrují přenášená data samotná, ale pouze klíče pro~symetrickou kryptografii.

\begin{table}[ht]
\begin{tabular}{ll}
Vygenerování privátních prvočísel & $p, q$ \\
Vypočítání privátního čísla & $\phi(n) = (p-1)(q-1)$ \\
Vypočítání veřejného čísla & $n = r \cdot s$ \\
Vypočítání veřejného klíče & $k_{\mathrm{pub}} \in [1, n], \mathrm{GCD}(k_{\mathrm{pub}}, \phi(n)) = 1$ \\
Vypočítání privátního klíče & $k_{\mathrm{priv}} = k_{\mathrm{pub}}^{-1} \Mod \phi(n)$ \\
\end{tabular}
\end{table}

Parametry $p, q$ by měly být podobně velké, i~když s~trochu rozdílnou velikostí (pro~ztížení útoků). Jejich společné faktory by měly být co nejmenší, pokud to prvočísla nejsou. Místo funkce $\phi$ se v~praxi používá Carmichaelovu funkci $\lambda$, protože $\phi$ může generovat čísla větší než je třeba.

\begin{table}[ht]
\begin{tabular}{ll}
šifrování & $c = m^{k_{\mathrm{pub}}} \Mod n$ \\
dešifrování & $m = c^{k_{\mathrm{priv}}} \Mod n$ \\
podepisování & $\mathrm{Sig}_m = m^{k_{\mathrm{priv}}} \Mod n$ \\
ověření & $\mathrm{Sig}_m^{k_{\mathrm{priv}}} \stackrel{?}{=} m \Mod n$ \\
\end{tabular}
\end{table}

\subsection[Diffie--Hellman výměna]{\href{https://en.wikipedia.org/wiki/Diffie-Hellman_key_exchange}{Diffie-Hellman výměna}}

Veřejnými prvky jsou prvočíslo $p$ (s~velikostí nad~2048~b), prvočíslo $q$ (větší než~224~b) a~generátor $z \in \mathbb{Z}_p^*$ velikosti $q$. Jde o~systém náchylný na~MitM, klíče nejsou nijak autentizované.

\begin{table}[ht]
\begin{tabular}{lcl}
Alice && Bob \\
$1 \le a \le p-2$ && $1 \le b \le p-2$ \\
$A = g^a \Mod p$ & $\stackrel{A}{\rightarrow}$ $\stackrel{B}{\leftarrow}$ & $B = g^b \Mod p$ \\
$k = B^a \Mod p$ && $k = A^b \Mod p$ \\
\end{tabular}
\end{table}

\subsection{Diffie--Hellman výměna nad~eliptickými křivkami}

Vlastnosti ECDH jsou totožné s~DH, DLP je pouze nahrazen jeho verzí nad~křivkami (ECDLP). Násobení na~eliptických křivkách je aplikováno jako opakované přičítání bodu. Asymetrické kryptosystémy nad~eliptickými křivkami nabízí stejnou bezpečnost při~mnohem kratší délce klíče (3072~b DH $\approx$ 160~b ECDH).

\begin{table}[ht]
\begin{tabular}{lcl}
Alice && Bob \\
$a$ && $b$ \\
$A = aP \Mod p$ & $\stackrel{A}{\rightarrow}$ $\stackrel{B}{\leftarrow}$ & $B = bP \Mod p$ \\
$k = aB \Mod p$ && $k = bA \Mod p$ \\
\end{tabular}
\end{table}

\subsection{Digitální podpisy}

RSA, ElGamal, DSA, ECDSA, ověření certifikátu, blockchain, PKCS\#1, X.509, \dots

\begin{center}
{\huge \dots} zde je třeba doplnit zbytek {\huge \dots} \\
(vlastně sám nevím co by tu přesně mělo být)
\end{center}

\clearpage
\section{Hašovací funkce -- vlastnosti, princip, použití, kolize, odolnost proti kolizím, příklady.}

\newpage
\section{PKI -- certifikát X.509 struktura, certifikační autorita základní části, časová razítka, autorita časových razítek.}

\newpage
\section{Generování náhodných čísel -- kryptografické generátory, požadavky, použití, princip, testování generátorů.}

Hlavními požadavky na~RNG jsou \textbf{rovnoměrné rozložení} (hodnoty jsou generovány se~stejnou pravděpodobností), \textbf{nezávislost hodnot} (neexistuje korelace mezi nimi), \textbf{nepredikovatelnost} (na~základě znalosti čísla nebo řady čísel nelze předpovědět následující, tzv. \emph{next-bit test}) a~často také \textbf{rychlé generování} nebo odolnost vůči kompromitaci (při~zjištění vnitřního stavu nelze zpětně rekonstruovat dosavadní vygenerovanou posloupnost, tzv. \emph{state compromise}). \emph{Next-bit} ochrana a~zábrana kompromitaci jsou požadavky kryptograficky bezpečných generátorů.

K~vytvoření dostatečného počtu (pseudo)náhodných čísel je nutná dostatečná entropie (míra náhodnosti/nepředvídatelnosti). Když útočník následující hodnotu zná, entropie je nulová. Lze ji vyjádřit jako $H(X) = - \sum_{i=1}^{n} p_i \log_2 p_i$ kde $X$ je generovaná hodnota, $p_1, \dots, p_n$ jsou pravděpodobnosti hodnot $X_1, \dots, X_n$ které je generátor schopen vygenerovat.

\subsection{Typy generátorů}

\subsubsection*{Fyzikální generátory (TRNG)}

Teoreticky jsou nedeterministické, nejsou ale známy přesné parametry, kterými by se daly popsat a~modelovat. Zaručují maximální bezpečnost a~neopakovatelnost. Nevýhodou je pomalé generování či~problematická technická realizace. Příklady: radioaktivní rozpad, tepelný šum, kvantové generátory.

\subsubsection*{Algoritmické generátory (PRNG)}

Posloupnost náhodná není, pokud útočníkovi nejsou známy některé parametry generátoru. Výhodami je rychlost, nízká odchylka od~poměru 1:1 nebo snadná realizace. Nevýhodami jsou bezpečnost či~periodicita. Příklady: čas, teplota HW komponent, šum na~\emph{low-level} sběrnicích (USB, pohyb myši), pohyb HDD.

\subsubsection*{Smíšené generátory}

TRNG z~fyzikálního generátoru se~spojí s~PRNG výstupem pomocí XOR.

\subsection{Příklady realizace PRNG}

\textbf{Bloková šifra v~režimu čítače}: náhodně se zvolí klíč a~počáteční hodnota $i$; zvoleným klíčem se šifrují hodnoty $i, i+1, \dots$ -- perioda u~$n$-bitové šifry je $2^n$. \\
\textbf{Hashovací funkce aplikovaná na~čítač}: hashuje se $i, i+1, \dots$. \\
\textbf{Proudové šifry}: viz otázku \ref{question-4}.

\subsection{Konkrétní příklady}

\textbf{\href{https://en.wikipedia.org/wiki/Blum_Blum_Shub}{Blum-Blum-Shub PRNG}} má formu $x_{n+1} = x_n^2 \Mod M$ (kde $M = pq$ je násobek dvou velkých prvočísel), výstup je~získán pomocí bitové parity výsledku, nebo z~jednoho či více nejméně významých bitů. Byl navrhnut v~roce 1986. \emph{Seed} by mělo být číslo nesoudělné s~$M$, s~vyloučením $\{0, 1\}$.

\textbf{Linuxový PRNG} sbírá události (myš, klávesnice, HDD, síť, \emph{system interrupts}, systémový čas). Zdroj \texttt{/dev/random} je blokující (při~nedostatku entropie se~čeká), na~rozdíl od~\texttt{/dev/urandom} (\emph{unblocked random}), kdy je vráceno \enquote{méně náhodné číslo}.

\textbf{EGD (Entropy Gathering Daemon)} využívají některé linuxové programy (OpenSSL, GPG, Apache HTTP) pokud není dostatečné množství náhodných bitů v~\texttt{/dev/random}. Entropii sbírá ve~stavu CPU, IO nebo sítě.

\subsection{Testování}

K~ověření vlastností PRNG se využívají statistické testy. Odhalí nekvalitní PRNG, ale neprokážou kvalitní PRNG. Testy vrací $p$-hodnotu, která vyjadřuje sílu důkazů proti nulové hypotéze (\enquote{Testovaná posloupnost je nádhodná.}). Pokdu $p$-hodnota překročí určitou mez, nulovou hypotézu považujeme za~neplatnou.

\textbf{Frekvenční test}: Obsahuje testovaná posloupnost bitů přibližně stejný počet nul a~jedniček? \textbf{Runs test}: Je počet a~délka řetězců po~sobě jdoucích stejných bitů na~úrovni náhodné posloupnosti? \textbf{Test hodností matic}, \textbf{spektrální test}.

\newpage
\section{Bezpečnostní architektura RM OSI -- služby bezpečnosti, mechanizmy bezpečnosti, útoky na bezpečnost, příklady implementace bezpečnostních mechanizmů v jednotlivých vrstvách.}
