\section{Prvočísla -- popište způsob generování a uveďte příklad pravděpodobnostního testu a skutečného testu}

\newpage
\section{Teorie čísel, algebraické struktury -- popište účel a~způsob výpočtu Eulerovy funkce. Popište požadavky na grupu a způsob generování grupy pro algoritmy založené na problému DL.}

\newpage
\section{Modulární aritmetika -- popište algoritmus Square and Multiply a Čínskou větu o zbytcích.}

\newpage
\section{Symetrická kryptografie -- proudové šifry, synchronní a asynchronní proudové šifry.}

\newpage
\section{Blokové šifry -- Product Ciphers, konstrukce, Feistelova síť, DES, AES, základní módy blokových šifer.}

\newpage
\section{Asymetrické algoritmy -- RSA, Diffie-Hellman, ECDH systém a jejich využití pro digitální podpis.}

\newpage
\section{Hašovací funkce -- vlastnosti, princip, použití, kolize, odolnost proti kolizím, příklady.}

\newpage
\section{PKI -- certifikát X.509 struktura, certifikační autorita základní části, časová razítka, autorita časových razítek.}

\newpage
\section{Generování náhodných čísel -- kryptografické generátory, požadavky, použití, princip, testování generátorů.}

\newpage
\section{Bezpečnostní architektura RM OSI -- služby bezpečnosti, mechanizmy bezpečnosti, útoky na bezpečnost, příklady implementace bezpečnostních mechanizmů v jednotlivých vrstvách.}
