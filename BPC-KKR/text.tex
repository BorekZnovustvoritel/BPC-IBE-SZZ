\section{Vzájemný vztah pojmů kyberkriminalita, kybernetická bezpečnost a~kybernetická obrana.}
Liší se v tom KDO CO provádí za JAKÝM účelem a za pomoci JAKÝCH prostředků.

\subsection{Kybernetická bezpečnost}
\begin{itemize}
    \item Kdo: CSIRT týmy (v ČR: Národní a vládní CERT tým)
    \item Co: Monitoruje počítačové infrastruktury.
    \item Účel: Cílem je zajistit zranitelnosti/kybernetické incidenty/CIA triádu (dostupnost, integrita, důvěryhodnost informací). \item Prostředky: technické, právní a organizační
    \item Právní úprava: Zákon č. 181/2014 Sb., o kybernetické bezpečnosti, Směrnice Evropského parlamentu a Rady (EU) 2016/1148 ze dne 6. července 2016 - Směrnice NIS, Směrnice o útocích na informační systémy, Vyhláška č. 82/2018 Sb., o kybernetické bezpečnosti a další vyhlášky a nařízení
\end{itemize}

\subsection{Kyberkriminalita}
\begin{itemize}
    \item Kdo: orgány činné v trestném řízení - policie, státní zastupitelství a soudy
    \item Co: Zde jsou dva možné pohledy. Prvním je trestná činnost proti CIA triádě. Druhým je kombinace prvního pohledu s obecně trestnou činností páchanou na internetu (podvod, dětská pornografie,...)  a s trestnou činností, kde se využívá informačních a komunikačních technologií.
    \item Účel: Ochrana před trestnou činností. 
    \item Prostředky: procesní postupy definované trestním řádem (zajišťování důkazů pomocí odposlechu, zajišťování dat, forenzní analýza,...)
    \item Právní úprava: Zákon č. 40/2009 Sb., trestní zákoník, Zákon o elektronických komunikacích č. 127/2005 Sb., Úmluva o kyberkriminalitě, Evropská úmluva o vzájemné pomoci ve věcech trestních č. 550/1992 Sb. (mezinárodní trestní právo),  Zákon o mezinárodní justiční spolupráci ve věcech trestních č. 104/2013 Sb. (mezinárodní trestní právo)
\end{itemize}

\subsection{Kybernetická obrana}
\begin{itemize}
    \item Kdo: resort Ministerstva obrany, Vojenské zpravodajství
    \item Co: Obrana proti kybernetickým útokům, které ohrožují suverenitu státu.
    \item Účel: Účelem je ubránit národní infrastrukturu.
    \item Právní úprava: Zákon č. 289/2005 Sb. Zákon o Vojenském zpravodajství
\end{itemize}



\newpage
\section{Prameny práva (národní, evropské i mezinárodní) obsahují hmotněprávní a procesněprávní úpravy kyberkriminality.}

\subsection{Mezinárodní}
\begin{itemize}
    \item Úmluva o počítačové kriminalitě
    \begin{itemize}
        \item 65 zemí z celého světa, “nejkompletnější mezinárodní standard dneška”, od 2001
        \item kapitola 1: základní pojmy, kapitola 2: trestní právo hmotné (klíčová část úmluvy) a procesní právo (získávání elektronických důkazů), kapitola 3: problematika mezinárodní spolupráce, kapitola 4:  procedurální záležitosti (územní působnost, výhrady, řešení sporů,...)
    \end{itemize}
    \item Dodatkový protokol o kriminalizaci činů rasistické a xenofobní povahy spáchaných prostřednictvím počítačových systémů
\end{itemize}

\subsection{Evropské}

\begin{itemize}
    \item Strategie kybernetické bezpečnosti EU
    \item Směrnice o potírání podvodů v oblasti bezhotovostních platebních prostředků (z 2017)
    \item Směrnice o útocích na informační systémy (z 2013, definice čtyř skutkových podstat počítačových trestných činů)
\end{itemize}

\subsection{Národní}
\subsubsection{Hmotněprávní úpravy}
Trestní zákoník č. 40/2009 Sb. ve znění pozdějších předpisů -- popis trestního práva hmotného - obsahuje
\begin{itemize}
    \item obecnou část (obecná definice pojmů, co je trestný
čin, jaké jsou možnosti trestu, urhný x souhrný trest, polechčující okolností, promlčení, kdo může
být pachatelem, zavinění (umysl x nedbalost))
    \item zvláštní část, kde je popis jednotlivých
skutkových podstat trestných činnů - jsou definovány skutkové podstaty týkající se přimo
kyberkriminality - neopravněný přístup k poč. systému §230, příprava hack nástroju §231 - nebo
jsou součástí skutkových podstat týkajících se běžné trestné činosti - porušení tajemstí §182,
porušení autorského práva §270, dětská pornografie §192, nebezpečné pronásledování §354,
podvod §209 - v těchto případech se často jedná o kvalifikovanou skutkovou podstatu nebot ICT
umožňují snažší páchání daného TČ s větším dopadem.
\end{itemize} 


\subsubsection{Procesněprávní úpravy}
Trestní řád - popisuje trestní právo procesní - definuje samotný proces a jeho fáze, zúčastněné
osoby, obviňěný, poškozený, obhájce a jejích postavení v třestním řízení - definuje zásady/postupy
trestního řízení/stíhání (procesní nástroje) - obecná součinnost §8, freezing §7b, odposlech §88,
zajištění provozních a lokalizačních údajů §88a odst. 1, sledování osob a věcí §158d - některé
nástroje jsou přímo spjaty s ICT (freezing, metadat), jiné jsou nejsou původně zamýšleny pro ICT
ale jsou tak nyní používané (sledování osob a věcí, pro získání dat od ISP).

Další prameny práva upravy kyberkriminality:
\begin{itemize}
    \item Zákon o elektronických komunikacích č. 127/2005 Sb
        \begin{itemize}
            \item Data retention -- uchování provozních a lokalizačních údajů po dobu 6 mesíců
        \end{itemize}
    \item Evropská úmluva o vzájemné pomoci ve věcech trestních č.
    550/1992 Sb. (mezinárodní trestní právo)
    \item Zákon o mezinárodní justiční spolupráci ve věcech trestních
    č. 104/2013 Sb. (mezinárodní trestní právo)
    \begin{itemize}
            \item definuje justiční spolupráci MLA - \textbf{Evropský vyšetřovací příkaz} (založen na uznávání příkazů - MR)
        \end{itemize}
\end{itemize}


\newpage
\section{Úmluva o kyberkriminalitě a směrnice o útocích na informační systémy (obsah úpravy a vztah k české právní úpravě)}

\textbf{Úmluva o kyberkriminalitě} 
\begin{itemize}
    \item nejúspěšnější instrument mezinárodního veřejného práva v oblasti kyberkriminality
    \item vydána v roce 2001, ČR podepsala v roce 2005, ratifikace proběhla v roce 2013
    \item OBSAH: kapitola 1: základní pojmy, kapitola 2: trestní právo hmotné (klíčová část úmluvy) a procesní právo (získávání elektronických důkazů), kapitola 3: problematika mezinárodní spolupráce, kapitola 4:  procedurální záležitosti (územní působnost, výhrady, řešení sporů,...)
    \item první dodatkový protokol: Dodatkový protokol o kriminalizaci činů rasistické a xenofobní povahy spáchaných prostřednictvím počítačových systémů, pracuje se na dalším (bude se týkat mezinárodní spolupráce)
    \item problémem je, že je z roku 2001, tudíž nereflektuje aktuální stav
    \item v právním řádu jednotlivých států musí být zajištěné:
    \begin{itemize}
        \item urychlené uchování dat (freezing) - TŘ §7b
        \item  urychlené uchování a vydání provozních a lokalizačních údajů - má existovat vydávací příkaz -
        možnost prohledání a zajištění dat - v Zákoně o elektornických komikacích
        \item vydání věci - trestní zákoník
        \item  možnost odposlechu komunikace - TŘ §88
        \item musí existovat orgán, který dokáže ve dne v noci 24/7 tuto spolupráci realizovat
        \item  mechanizmus pro dobrovolné předávání informací - dále je v umluvě upraven způsob využití
        těchto procesních nástrojů - stále je to všechno postaveno na MLA - mezinárodní justiční
        spolupráci - úprava v trestním řádu
        \item možnost přímého přístupu k datům v zahraničí při souhlasu
    \end{itemize}
    \item Brazílie a Indie odmítly úmluvu přijmout s odůvodněním, že se nepodílely na jejím vypracování, Rusko se staví proti Úmluvě s tím, že její přijetí by porušilo ruskou suverenitu
\end{itemize}
\textbf{Směrnice o útocích na informační
systémy}
\begin{itemize}
    \item skutkové podstaty jsou formulované v podstatě stejně jako v Úmluvě o kyberkriminalitě
\end{itemize}

\newpage
\section{Postupy a kriteria při kvalifikaci trestné činnosti (vč. problematiky kvalifikované a privilegované skutkové podstaty)}

\subsection{Postupy při kvalifikaci trestné činnosti}

Podnět → přípravné fáze → rekognoskační fáze (jde opravdu o TČ?) → zahájení vyšetřování → vyšetřování el. místa činu → vyšetřování fyzického místa činu → zpracování důkazů → dokazování před soudem.
\begin{itemize}
    \item První fází je detekce trestného činu. To může být buď nahlášení oběti, nebo zjištění trestného činu prostřednictvím monitorovacích systémů, bezpečnostních auditů a dalších metod. Dále se prověřuje, zda se opravdu jedná o trestný čin. V této fázi mohou hrát roli policie, bezpečnostní specialisté a další.
    \item Další fází je identifikace pachatele. To může být obtížné v případě kybernetických trestných činů, protože pachatelé často používají anonymizační technologie a další taktiky k utajení své totožnosti. V této fázi mohou pomoci kybernetičtí specialisté, kteří provedou analýzu digitálních stop a dalších technických prvků, aby identifikovali pachatele.
    \item Zárověň probíhá sběr důkazů, což je klíčovou fází stíhání kybernetických trestných činů. To může zahrnovat sběr dat z počítačů a serverů, záznamů o síťovém provozu a další. V této fázi mohou pomoci kybernetičtí specialisté, kteří mají technické znalosti potřebné k extrakci a analýze digitálních důkazů.
    \item  Po identifikaci pachatele a shromáždění dostatečných důkazů může být provedeno zatčení. V této fázi hrají roli orgány činné v trestním řízení, jako je policie a státní zastupitelství.
    \item Po zatčení pachatele a shromáždění dostatečných důkazů následuje soudní proces. V této fázi hrají roli soudy, soudní znalci a advokáti, kteří se podílejí na soudním procesu.
\end{itemize}

\subsection{Kritéria při kvalifikaci trestné činnosti}

\begin{itemize}
    \item zjištění zda jde o TČ nebo ne = zjištění jestli skutkové okolnosti naplňují formální znaky skutkové TČ pospsaného v TZ ve zvláštní části
    \item okolnosti vyloučení protiprávnosti - některé TČ jsou trestné pouze za určitých okolností (místo, čas, způsob) \item promlčení
    \item odpovědnost podezřelého -- věk, příčetnost
    \item určení zavinění - úmysl a nedbalost
    \item zohlednění účinnosti Trestního práva - zásady jusrisdikce - teriotrialita, registrace, personalita, ochranná a univerzální
    \item zjištění zda jeden spáchaný skutek nenaplňuje znaky i dálších trestných činnů, subsumpce pod více TČ
    \item souběh TČ a jejich vyloučení - pachatel nesmí být potrestnán dvakrát za jeden čin
    \item určení míry trestu -- kvalifikovaná skutková podstata, polechčující přitěžující okolnosti, úhrný
    a souhrný trest
\end{itemize}

\subsubsection{Úhrný a souhrnný trest}
Společným znakem úhrnného trestu a souhrnného trestu je to, že jde o tresty ukládané za dva nebo více trestných činů, které byly spáchány v souběhu, a že jsou ukládány podle týchž zásad. Rozdíl záleží v tom, že úhrnný trest se ukládá za sbíhající se trestné činy, kterými byl pachatel uznán vinným jedním rozsudkem ve společném řízení, zatímco souhrnný trest se ukládá za situace, kdy ohledně některého ze sbíhajících se trestných činů již bylo rozhodnuto výrokem o vině a trestu a kdy se pro ostatní ze sbíhajících se trestných činů vede samostatné řízení.

\subsection{Problematiky kvalifikované a privilegované skutkové podstaty}
Skutková podstata je souhrn typických, základních, právně relevantních znaků určitého právního institutu (například: trestného činu). Pokud osoba naplní svým jednáním takové znaky (například spáchá trestný čin), naplní skutkovou podstatu, což s sebou přináší právní následky stanovené příslušnou právní normou.

Základním tříděním skutkových podstat v trestním právu je třídění na skutkové podstaty základní, kvalifikované (přísněji trestné kvůli vyšší nebezpečnosti jednání) a privilegované (výjimečně mírněji trestné kvůli nižší nebezpečnosti). U dvou posledně jmenovaných přistoupily ke znakům základní skutkové podstaty některé právně relevantní znaky další.

\subsubsection{Obligatorní znaky skutkové podstaty}
\begin{itemize}
    \item objekt (právní statek, který je jednáním porušen či ohrožen)
    \item objektivní stránka (jednání - konání či opominutí a příčinná souvislost mezi jednáním a následkem)
    \item subjekt (pachatel)
    \item subjektivní stránka (vnitřní psychický vztah pachatele k jeho protiprávnímu chování a zavinění - úmysl či nedbalost, příp. někdy je stanovena i pohnutka)
\end{itemize}




\newpage
\section{Kategorizace kyberkriminality (včetně příkladů trestné činnosti v jednotlivých kategoriích)}

\subsection{Prostřednictvím klasifikace kybernetických bezpečnostních incidentů} 
Detailnější třízení, v souladu s realitou, neváže se na skutkové podstaty. Kategorie: sběr informací, škodlivý kód, dostupnost, pokus o průnik, průnik,  informační bezpečnost, podvod, škodlivý obsah.

\textbf{Sběr informací} - scanning, sniffing (odposlouchávání protokolů), phising. \textbf{Škodlivý kód} - virus, torjan, spyware, distribuce (škodlivého kódu), Command and control (systém, ze kterého je řízeno fungování sítě zařízení infikovaných škodlivým systémem). \textbf{Dostupnost} - Dos, DDos, sabotáž. \textbf{Pokus o průnik} - využívání zranitelností, pokus o přihlášení. \textbf{Průnik} - využívání zranitelností, zneužití účtu. \textbf{Informační bezpečnost} - neautorizovaný přístup, neautorizovaná modifikace/smazání. \textbf{Podvod} - zneužití nebo neautorizované využití zdrojů, neoprávněné využití jména třetí strany. \textbf{Škodlivý obsah} - spam, duševní vlastnictví (protiprávní užívání), dětská pornografie, rasismus, schvalování násilí.

\subsection{Další možnosti kategorizace}
Podle trestního práva hmotného - skutkové podstaty, menší granularita, problém se statistikou.

Podle UNODC (United Nations Office on Drugs and Crime) - útoky na CIA triádu, trestné činy související s počítačem (spamming, vydíraní,...), trestné činy související s obsahem (dětská pornografie).

Podle využití ICT prostředků
\begin{itemize}
    \item \textbf{Cyber-dependent} - páchané jedině prostřednictvím ICT prostředků, příklad: Neoprávněný přístup k počítačovému systému a nosiči informací (§230 TZ), Neoprávněné opatření, padělání a pozměnění platebního prostředku (§234 TZ)
    \begin{itemize}
        \item ohrožující CIA triádu
        \item využívající ICT
    \end{itemize}   
    \item \textbf{Cyber-enabled} - klasické činy páchané prostřednictvím ICT prostředků, příklad: Výroba a jiné nakládání s dětskou pornografii (§192 TZ), Nebezpečné pronásledování (§354 TZ), Porušení autorského práva (§270 TZ)
    \item \textbf{Cyber-supported} - incidentální využití ICT, byli použity ICT a tak vznikají el. důkazy (telefonování)
\end{itemize}

\newpage
\section[Kyberkriminalita v užším smyslu slova (příklady trestné činnosti a~kvalifikace dle zvláštní části TZ)]{Kyberkriminalita v užším smyslu slova (příklady trest-né činnosti a kvalifikace dle zvláštní části TZ)}
\textbf{Kyberkriminalita} -- v užším smyslu do něj zahrnujeme \textbf{trestnou činnost}, která směřuje právě přímo proti důvěrnosti, dostupnosti, či integritě informačních systémů (neoprávněný přístup k počítačovému systému a nosiči informací, nebo opatření a přechovávání hesla a přístupového zařízení k počítačovému systému).

\subsection{Příklady trestné činnosti}
\begin{itemize}
    \item \textbf{Scanning} - pouze pokud je to prováděno na účelem následného zneužití (§230, §182) - Opatření a
přechovávání přístupového zařízení a hesla k počítačovému systému a jiných takových dat §231
    \item \textbf{Sniffing} - Poručení tajemství dopravovaných zpráv §182 - Příprava nástrojů §231 - Neoprávněné
opatření, padělání a pozměnění platbního prostředku §234
    \item \textbf{Phishing} - Příprava nástrojů §231, získání např přístupových údajů za účelem §230 nebo §182 -
Podvod §209 - Platební prostředky §234
    \item \textbf{Ransomware} - Podvod §209 - Neoprávněný přístup §230 - Porušení tajemství listin a jiných
dokumentů uchovávaných v soukromí §183
    \item \textbf{(D)DoS} - Neoprávněný přístup §230 - jsou tam na to speciální písmena - v DDOS ještě dálší §230
aby získal botnet - teoreticky i §231 příprava nástoje pro následující útok
    \item \textbf{Využivání zranitelnosti/exploitu} - §230 při dokonání - §231 při neúspěšném pokusu je stále
dokonaná příprava (musel si ten exploit připravit) takže kvalifikace podle §231
\end{itemize}


\newpage
\section{Elektronické důkazy a jejich specifika v trestním řízení.}
Základní specifika/problémy:
\begin{itemize}
    \item neexistuje specifická úprava pro práci s el. důkazy → používají se standardní nástroje TP procesního
    \item využívá se znalců v oboru
    \item nutnost přenesení důkazu do vnímatelného zachycení, aby se sním mohl seznámit soudce, el. důkazy se těžko převádí do vnímatelného zachycení (dokazování zdrojovým kódem nebo metadaty se nedá jentak samo použít → nutnost interpetace)
    \item nedostatek obecně uznávaných postupů a judikatury (soudy moc ještě nerozhodovali o specifikách el. důkazů), OČTŘ nemá jasně definovaný postup jak s el. důkazy nakládat, na různých úrovních nebo v různých krajích se to dělá jinak
    \item volatilita el. důkazů = data mohou mizet (freezing)
    \item často přeshraniční charakter
    \item nedostatek vzděláných lidí, znalci, zástupci, soudci 
\end{itemize}

\textbf{Základní zásady dokazování = presumpce neviny, zásada ústnosti} (ústně a prezenčně se
důkazy předávají soudci a všem okolo, problém s el. důkazy),  \textbf{zásada veřejnosti, zásada bezprostřednosti} (soudce je seznámen s důkazy bezprostředně, problém s el. důkazy, ne
všechno se dá přinést k soudu, potřeba vymyslet mechanismus jak soudce s důkazem seznámit), \textbf{zásada materiální pravdy, zásada vyhledávací, zásada volného
hodnocení důkazů} (neexistuje dělení důležitosti důkazů, vždycky je to subjektivní hodnocení,
problém s el. dukazy, když soudci někdo pořádně důkaz nevysvětlí je pro něj velmi těžké důkaz hodnotit a tak může i velmi kvalitní el. důkaz diskreditovat), \textbf{zásada zdrženlivosti} (redukce škod, například při zajišťování důkazů při domovní prohlídce)

\subsection{Důkaz}
Jakákoliv informace, kterou získáváme, vyvrací nebo potvrzuje skutkovou okolnost, může to být cokoliv. Musí být získán \textbf{zákonným postupem} (získán důkazním prostředkem).
Dokazuje se: 
\begin{itemize}
    \item zda se stal skutek, který považujeme za TČ, kdo ho spáchal (zda to byl obviněný)
    \item z jakých pohnutek (důvodu), aby se dalo rozhodnout jestli skutek nenaplňuje kvalifikovanou skutkovou podstatu
    \item následky TČ (co bylo napácháno za škody a že je mezi nimi kauzální nexus), poměry obviněného, okolnosti vedoucí ke spáchání nebo okolnosti (spolupachatelné, nedbalost apod)
\end{itemize}
\newpage

\subsection{Zajištění elektronických důkazů}
Trestní řád neobsahuje výslovnou úpravu, a je tedy třeba k zajišťování elektronických důkazních prostředků často využívat nepříliš vhodné procesní nástroje. K počítačovým datům se lze dostat v zásadě třemi základními způsoby:
\begin{itemize}
    \item Zajištění zařízení či datových nosičů (hardware) -- Vydání a odnětí věci, osobní prohlídka, domovní prohlídka a prohlídka jiných prostor.
    \item Přímý přístup k datům (lokální, vzdálený)
    \begin{itemize}
        \item orgány můžou přístoupit k lokálním datům přímo bez souhlasu, pokud jsou data součástí odejmuté věci
        \item orgány můžou přístoupit k vzdáleným datům přímo bez souhlasu, pokud znají přístupové údaje (našli je napsané nebo od manželky), potřebný souhlas soudu
        \item k datům na připojených službách:
        \begin{itemize}
            \item podle postupu §158 odst. 3  (Sledování osob a věcí) -- aktuální data (nutné pořídit protokol k dokazování)
            \item podle postupu §88 (Odposlech) -- budoucí data
        \end{itemize}
        V obou případech je taktéž potřebný souhlas soudu.
    \end{itemize}
    \item Přístup prostřednictvím držitele/správce dat -- požádání toho, kdo je má v držení nebo je spravuje, aby mi ty data zpřístupnil nebo dal (data retention).
\end{itemize}

\newpage
\section{Procesní nástroje pro zajišťování elektronických důkazů.}

\subsection{Obecná součinnost §8}
Státní orgány, fyzické a právní osoby a dálší relevantní subjekty mají povinnost vyhovovat na
dožádání - OČTŘ chce informaci a tak provede dožádání a daný subjekt by na tuto žádost měl
vyhovět - pokud existuje specifická právní úprava musí se použít ta - plus soudy pak řeší
proporcionalitu se zásahem do práv suběktu - zjišťování nejzákladnějších informací (obvykle v
rekognoskační fázy) - pro utajované informace je potřeba příkaz(souhlas) soudce (§8/5 "paragraf 8
odstavec 5)
\subsection{Freezing §7b}
Vyžaduje to po nás Úmluva o kyberkriminalitě - 2 procesní nástroje
\begin{itemize}
    \item \textit{freezing} -- hrozí ztráta zníčení nebo pozměnění dat důležitých pro třestní řízení, lze nařídít osobě která je drží, aby je uchovala v nezměněné podobě pro potřeby dálšího vydání vyšetřovatelům (při phishingu zamrznutí nasbírané databáze)
    \item \textit{blocking} -- příkaz na provozovatele služby, aby zablokoval přístup užívatele k daný datům (max 90 dnů), poskytovately služby (ten kdo freezing provádí) musí být vysvětleno co má být zmraženo, proč a na jak dlouhou dobu
\end{itemize}

\subsection{Odposlech a záznam telekomunikačního provozu §88}
při vedení TŘ pro zločin s odnětím svobody min. 8 let nebo pro vyjmenované TČ nebo pro
umyslné TČ k jehož stíhání nás zavazuje mezinárodní smlouva
\begin{itemize}
    \item může být vydán příkaz (úkon) pro zajištění obsahu telekomunikačního provozu (tekoucí data, e-maily telefonáty) - zajištěno \textbf{Útvarem zvláštních činností} 
    \begin{itemize}
        \item mívají nainstalované zařízení, které umožnuje tento odposlech
        \item odposlech realizují ve spolupráci s operátorama a ti za to dostávají peníze
    \end{itemize}
    \item pokud nelze sledovaného účelu dosáhnout jinak nebo pokud by to bylo moc složité (preferují se
jiné úkony) - potřeba příkaz soudu - nebo i bez soudu se souhlasem uživatele - odposlech možný
vydat i dobudoucna (i když OČTŘ neví zda TČ probíhá) musí to však být stále dobře odůvodněné
\end{itemize}

\subsection{Zajištění provozních a lokalizačních údajů §88a odst. 1}
metadata k tekoucí komunikaci - na základe příkazu soudu - pokud úmyslný TČ min 3 roky nebo
vyjmenované nebo TČ vyhlášený mezinározní smlouvou - pokud učelu nelze dosáhnout jinak -
nutno zajistit aby tyto údaje byly uloženy poskytovatelem - na to máme úpravu Zákona o
elektronických komunikacích

\subsection{Data retention §97 - Zákon o elektronické komunikaci (ZoEK)}
ZoEK říká co jsou provozní a lokalizační údaje:
\begin{itemize}
    \item \textbf{metadata o komunikaci} - údaje by neměli nic říkat o přenášených datech (poskytovatel by třeba měl odfiltrovat data obsažená v URL třeba z formulářů)
    \item  poskytovatel je na základě data retention povinen uchovávat provozní a lokalizační údaje po dobu 6 měsíců od doby uskutečnení daného komunikačního provozu (za to poskytovatel dostává peníze), nesmí být poskytovatelem zneužita. Pokud nejsou po 6 měsících vyšetřovately požadovana musí je tekomunikační operátor zkartovat
\end{itemize}

\subsection{Sledování osob a věcí 158d trestního řádu}
Pátrací prostředek - primárně určen pro zajištění operativních informací -> zjistit co se
stalo, ale jé na zjištění důkazů pro soud (tak tomu bylo původně)

Pokud chci zjištěné informace použít jako důkaz u soudu musím o tom vypracovat protokol - dnes se to používá i pro
zajištění el. dat jako důkazů. Postup:
\begin{enumerate}
    \item vyšetřovatel potřebuje data z uložiště -> zahájím sledování osob a věcí
    \item jako součinnost si vyžádám spolupráci od poskytovatele služby, který data uchovává -> a v rámci té součinnosti mi poskytně i ty data, která chci
    \item když udělám protokol o tom jak jsme získal danou spolupráci, kdo mě poskytnul součinnost, jak jsem postupoval a jaká data jsem získal -> mužu daná data použít jako el. důkaz u soudu
\end{enumerate}
O vydání příkazu o sledování osob a věcí rozhoduje státní zástupce prostřednictvím povolení - pokud jsou to ale soukromá nebo utajovaná data je potřeba předchozí povolení soudce - dá se udělat neodkladný úkol (získám telefon a neodkladně se kouknu na data uložená na vzdálené službě) pokud pětně bude soud souhlasit.
\newpage
\subsection{Domovní prohlídka a prohlídka jiných prostor}
Postup:
\begin{itemize}
    \item Soudce vydá příkaz k realizaci domovní prohlídky
    \item  policejní orgán tam provádí ohledání věcí relevantní k trestnímu stíhání - policejní orgán musí soudci dostatečně vysvětli proč se domnívá že v příslušných prostorách jsou veci důležité pro TŘ
    \item soudní zástupce sepíše podání na soud ve kterém žádá o vydání příkazu k domovní prohlídce ve které uvede co je cílem a proč a proč si myslím že to tam bude a oduvodním že to nejde jiným nástrojem
    \item  soud to posoudí a v rámci rozhodnutí pak vypíše informace na základě kterých se rozhodoval a tím oduvodní vydání daného příkazu - nedá se odvolat, ale stát ručí za škody způsobené nesprávným vydáním příkazu o domovní prohlídce
\end{itemize}
Procesní podmínky za kterých může být domovní prohlídka realizovaná:
\begin{itemize}
    \item přiměřenost -  odborná péče buď vyšetřovatel nebo znalec (když to dokážou ohledat na místě nemusí se to odvážet)
    \item potřeba přítomnosti majítele prostor nebo musí být aspoň informován o tom co se tam děje
    \item nezúčastněná osoba (někdo externí, soused) kdo zkontroluje že nedochází k porušení zákona při prohlídce
    \item vypracovává se protokol (videozáznam, foto) - všichni kdo se zučastní ho podepisují i nezúčastněná osoba
\end{itemize}
Aby přistihly zločince se zapnutím PC a přihlášením do vzdálené služby - dá se přizvat znalec pro specifickou
činnost se specifickým vybavením - musí se dodržovat specifické postupy aby nebyly důkazy znehodnocovány - zapečetění, zabalení do pytle za přítomnosti nezúčastněné osoby.

\subsection{Výdání a odnětí věci}
Nástroj pro zajištění věci od člověka - každý má Ediční povinnost (na požádání musí vydat drženou věc) - OČTŘ ho vyzvou k předložení věci - pokud odmítně věc vydat může mu být odejmuta (pořádkové opatřené) stačí rozhodnutí státního zástupce - při odejmutí věci by měla být přítomna nezúčastněná osoba - pořizuje se protokol - osobě se dá potvrzení o odejmutí - pokud jsou na zařízení data o kterých je povinná mlčenlivost (utajované informace, advokátní tajemství) je specifický postup - pouze věci né data - ovšem pokud je vydán např telefon tak naněm může být provedena forenzní analýza a tedy je možno se dostat k datům uloženým na zařízení

\subsection{Osobní prohlídka}
domnívám (jako OČTŘ) se že daná osoba má u sebe věc důležitou pro TŘ, ale nevím to jistě -> zahájím osobní prohlídku - na základě rozhodnutí soudu nebo státního zástupce - pokud je to neopakovatelný úkon můžu osobní prohlídku provéset i bez příkazu (zatknu osobu co utíká z místa činu a mám podezření že u sebe má zbraň se kterou páchal TČ) musím pak ale souhlas zpětně získat - je to zhojitelná vada neučinného důkazu - úkonu by měl předcházet předchozí výslech (jako u domovní prohlídky) což bych osobu měl požádat zda věč/předmět u sebe má a zda mi ho nevydá.

\subsection{Ohledání věci}
pozorování a sbírání informací za účelem objasnění věci - protokol co bylo vypozorováno a jak - typicky při domovních a osobních prohlídkách - na místě najdu spuštěný počítač a na místě chci provést jeho ohledání - v takovém případě s kamerou nebo foťákem provádím záznam toho co jsem objevil.

Nemůžu provést obecné ohledání věci - např. když ohledávám na místě zaplé PC a bude tam probíhající komunikace a já bych ji chtěl sledovat (odposlech má větší zásah do práv), tak to nemůžu udělat jen na základě ohledání věci (generovalo by to neúčinný důkaz) - avšak policie si proto může předem připravit příkazy- například existují příkazy k domovní prohlídce, odposlechu a sledování osob a věcí a tím pádem může získávat všechno na místě.

\newpage
\section{Specializované útvary OČTŘ ve vztahu ke kyberkriminalitě.}
\subsection{Policie ČR}
\begin{itemize}
    \item\textbf{Policejní presidium} → Skupina informační kriminality, hlavní organizační prvek Policie ČR, koordinace a metodika, cílení na konzistenci pri vyšetřování
    \item \textbf{Odbor kriminalistické techniky a expertíz (OKTE PČR)} → znalecký ústav, akreditace na digital forensics (získávání el. důzaků z dat), znalecká činost, málo lidí, tak tyto služby nabízí soukromnící.
    \item \textbf{Kriminalistický ústav Praha} → Oddělení počítačové expertízy, také znalecká činost ale jako externí služba
    \item \textbf{Útvar zvláštních činností Policie ČR (UZČ SKPV PČR)} → centrální útvar policie v Praze, expozitury v jednotlivých krajích, poskytuje vyšetřovatelům služby (odposlech a zajišťování dat od poskytovatelů služeb (provozní a lokalizační údaje), má k tomu kontaktní sítě a vybavení, vyšetřovatel požádá soud o příslušný příkaz, ten dá utvaru zvláštních čiností, který daný úkon provede (zajístí odposlech, zajístí data) a pak je v
    protokolu předává
    \item \textbf{Národní centrála proti organizovanému zločinu} → spojení útvaru pro odhalování organizovaného zločinu a útvaru pro odhalování finanční kriminality, vyšetřování nejzávažnějších druhů kriminality včetně rozsáhlých kybernetických útoků (rozsáhlé
    ransomwary, útoky na kritickou infrastrukturu), expozitury v jednotlivých krajích, 2 role (metodická a vyšetřovací)
    \item \textbf{Analogická pracoviště na krajských ředitelstvích PČR (oddělení kybernetické kriminality,
    odbor poč./inform. kriminality)} → různé jméno v různých krajích, vyšetřování na lokální úrovni, nejúspěšnější je Jihomoravský kraj, snaží se předávat znalosti do ostatních krajů
    \item \textbf{EC3 jednotka} → Europol, Interpol a další.
\end{itemize}

\subsection{Státní zastupitelství}
Neexistují zvláštní specifické instituce/zástupci, pouze neformální skupiny na úrovni nejvyššího
státního zastupitelství nebo na úrovni krajského státního zastupitelství, předávání metodiky a
snaha o zajištění koordinace a vzdělávání. Obecně velký problém nevzdělanosti (do úrovně justiční akademie), dělají to dobrovolně.
\subsubsection{Nejvyšší státní zastupitelství (NSZ)}
Významná role, vydává metodické pokyny.

\subsubsection{Justiční akademie}
Příspěvková organizace státu, poskytuje nadstandardní vzdělávání soudům a státním zastupitelstvím.

Školení NÚKIB, semináře NSZ, akademická sféra - Ústav pro kriminologii a sociální prevenci, úzká vazba na Ministerstvo spravedlnosti a na nejvyšší složky policie (zatím jen statistické studie).


\newpage
\section{Mezinárodní spolupráce v oblasti kyberkriminality.}
Působnost práva je podmíněna existencí státní moci schopné právo vymáhat a je teritoriálně omezena na principu suverenity (právo státu vykonávat státní moc nad určitým územím). Teritoriální charakter práva nesedí s charakterem informačních sítí - geolokace a georestrikce funguje spíše v soukromoprávních vztazích Fyzická poloha komunikujících stran není podstatná. Kybernetická kriminalita zpravidla zahrnuje přeshraniční prvek (pachatel a poškozený v jiných státech nebo pachatel a poškozený ve stejných státech, ale vetšina dat je v zahraničí). Někdy má zásadní vliv na způsob i úspěšnost vyšetřování a stíhání. Často je nutná součinnost definičních autorit a mezinárodní spolupráce. 

\subsection{Hlavní problémy}
Podle kterého práva? Co je a není trestný čin? Kdo a jak má získávat důkazy a provádět úkony? Jak spolupracovat?

\begin{itemize}
    \item neharmonizovaná legislativa (harmonizace je pomalá), využívání bezpečných přístavů, negativní kolize, kyberprostor nemá hranice ale právo ano
    \item neochota některých států spolupracovat nebo některé státy nejsou dostatečně vyspělé a nemají lidi na to aby spolupráci poskytli
    \item pomalá spolupráce -- pachatel se může rychle přemisťovat (i se serverem)
    \item neexistence nástroje pro el. předávání důkazů
\end{itemize}

\subsection{Podle kterého práva?}
\begin{itemize}
    \item \textbf{suverenita} = kompletní a exklusivní právo státu vykonávat státní moc nad určitým územím (často se uplatňuje i v rámci kyberprostoru, ale je problematické její vymezení např. podle lokalizaci infrastruktury)
    \item \textbf{jurisdikce} = právo státu definovat povinnosti a práva lidí v rámci svého teritoria, vymáhat pravidla a trestat jejich porušení
    \item \textbf{rozhodné právo} = právo podle kterého bude posuzována určitá právní skutečnost (ve veřejném právu platí lex fori -- nedělitelnost jurisdikce a rozhodného práva, v soukromém si lze zvolit)
\end{itemize}

Vymezení jurisdikce ČR na základě principů: 
\begin{enumerate}
    \item \textbf{princip teritoriality} = vztah k prostoru ČR → např. v jednom státě může být provozován malitious server, v druhém může být skupina lidí koordinující útok a ve třetím státě muže být dopad útoku (všechny tři státy si mohou nárokovat jurisdikci podle principu teritoriality)
    \item \textbf{princip personality} = aktivní (občan ČR spáchal) a pasivní (vůči občanu ČR byl ospácháno)
    \item \textbf{princip registrace} = letadla a plavidla pod vlajkou ČR spadají pod jurisdikci ČR
    \item \textbf{princip ochrany a univerzality} = obecné principy, okrajové principy (moc nenastávají) → třeba když někdo páchá genocidu, tak je to stíhatelné podle práva ČR vždycky, bez ohledu na to kde je to páchané, v případě požádání cizího státu, který na má připad v jurisdikci, aby byl případ odstíhán v ČR → bude se řešit podle českého práva
\end{enumerate}

Z těchto principů je vidět že jurisdikci si může nad určitým případem nárokovat více států najednou (kolize jurisdikcí). Může vzniknout pozitivní a negativní kolize. \textbf{Pozitivní kolize} je případ, kdy si více států nárokuje jurisdikci (právo státu definovat povinnosti a práva lidí). Naopak \textbf{negativní kolize} je případ, kdy si nikdo nenárokuje jurisdikci. Pachatel může distribuovat svou činnost tak, aby to co páchá nebylo v daném státu trestné (bezpečný přístav). Vznik kolize se řeší například domluvou nebo to řeší nějaká mezinárodní organizace (určí si, kdo bude stíhat nebo jestli založí společný vyšetřovací tým a budou spolupracovat). Také se používají mezinárodní a regionální smlouvy, dvoustranné úmluvy, vzorové právo a právo EU.

\subsection{Co je a není trestný čin?}
To stanovují jednotlivé státy v rámci své suverenity. Platí různé právní kultury, metody úpravy, hodnoty, různé definice trestných činů. Diskrepance mohou vést ke vzniku takzvaných bezpečných přístavů (neexistuje skutková podstata, nevymáhá se, nejsou nástroje k vymáhání, tresty jsou nízké) -> tendence k harmonizaci (někdy ale není možná). Bohužel se setkáváme se Složitým hledáním konsensu (OSN)

\subsection{Harmonizační nástroje}
Mezinárodní úmluvy
\begin{itemize}
    \item Úmluva o kyberkriminalitě (2001)
\end{itemize}
Regionální úmluvy
\begin{itemize}
    \item Dohoda o spolupráci při boji proti kriminalitě související s počítačovými daty (Commonwealth, 2001)
    \item Dohoda o boji proti IT trestných činech (Arabská liga, 2010)
    \item Dohoda o spolupráci v oblasti mezinárodní informační bezpečnosti (Shanghajská organizace spolupráce, 2010)
    \item Úmluva o kybernetické bezpečnosti a ochraně osobních údajů (Africká unie, 2014)
\end{itemize}
Dvoustranné úmluvy

Vzorové právo
\begin{itemize}
    \item Vzorové právo pro počítačovou kriminalitu a kyberkriminalitu (Jihoafrická společnost pro spolupráci)
\end{itemize}
Právo EU
\begin{itemize}
    \item Směrnice o útocích na informační systémy (2013, opisuje úmluvu)
\end{itemize}


\subsection{Snahy do budoucna}
OSN se snaží ale kvůli velikosti má problém něco prosadit, nepodařilo se příjmout ani studii o zmapování páchání
kyberkriminality po světě (zůstalo to ve formě návrhu). Státy se snaží rozšířít svou jurisdikci (především EU a USA, příkladem GDPR, které platí pro všechny na území EU). Druhá otázka je pak vymahatelnost těchto pravidel, EU se to líbí a začínají to uplatňovat i do ostatních úprav, USA
příjmulo CLOUD act, který říká, že je jedno kde máš data o amerických občanech, ale pokud je to v zahraničí tak má povinnost je vydat americkým úřadum. Do určité míty dochází k fragmentaci internetu (Rusko a Čína to velmi podporují). Čína si striktně dozoruje co jde dovnitř a ven (velký čínský firewall, např. data o čínskách občanech se nesmí uchovávat mino čínu). Rusko vyhrožuje totálním odpojením jejich části sítě.

\subsection{Procesní právo a spolupráce}
Omezená jurisdikce → omezené možnosti OČTŘ provádět úkony nebo získávat důkazy ze zahraničí. Často neochota nebo nemožnost ISP k dobrovolné spolupráci → nutná součinnost se zahraničním subjektem nebo autoritou. Jeden způsob je možnost stát obejít a rovnou řešit vše s poskytovately konkretních služeb - obecně mívají i více lidí, kteří mají ke spolupráci dostatek znalostí + většinou jsou to stejně oni, kdo drží data.


Formální
– MLA
– MR (EU)
– Dvoustranné nástroje
• Neformální
• Přímá s ISP
– Preservation and production směrnice (EU)
– Cloud act (US)
– dobrovolná
\begin{itemize}
    \item Formální součinnost
    \begin{itemize}
        \item MLA (Mutual Legal Assistance) mezinárodní justiční pomoc -- nejstandardnější mechanismus upravený v mezinárodních smlouvách a ve většině právních řádů, umožňuje OČTR sepsat žádost o mezinárodní justiční spolupráci která se předá pomocí komunikujících zastupitelských úřadů a čeká se než dožádaný stát poskytne spolupráci (velmi pomalé, stát nemusí vyhovět)
        \item MR (Mutual Recognition) vzájemné uznávání rozhodnutí -- fungující specificky v EU na základě Evropského vyšetřovacího příkazu, uplatňuje se u některých procesních nástrojů, které upravují evropské předpisy, dožádaný stát musí příjmout rozhodnutí jiného státu a s určitou prioritou ho musí vykonat
        \item Dvoustranné nástroje -- dva státy se dohodnou na užší spolupráci, většinou státy které mají k sobě blízko, se sousedními státy (když někoho stíhají v autě a přejedou hranice tak můžou pokračovat)
    \end{itemize}
    \item Neformální spolupráce států -- orgány nebo lidi se znají a tak si poskytnou spolupráci nebo se aspoň nasměrují na správnou cestu (konzultace, expertýzy, vybavení), může to koordinovat Europol nebo Interpol nebo může být zprostředkována přes bezpečnostní týmy nebo jiné státní orgány či sdružení (sdružení energetických společností), neformální cesty negenerují tak kvalitní důkazy, ale je to rychleji (pokud třeba jen zjišťujeme kde se důkazy mohou nacházet, tak se můžeme jednoduše neformálně doptat jiných států/orgánu/společností)    \item Přímá spolupráce s ISP -- Existující právní regulace → Preservation and production směrnice (EU) - Cloud act (US) nebo dobrovolná spolupráce
\end{itemize}

\subsection{Další aktuální iniciativa}
\begin{itemize}
    \item Příkaz k zachování a vydání dat -- zatím nepřijatá směrnice, přímá povinnost ISP v rámci EU vyhovět dožádání podle práva MS, související nařízení směřující k určení zástupce, rozšiřování vlivu práva EU, možná budoucí efektivní nástroj
    \item UNODC -- snažili se o vzorové právo, studii, která nakonec nebyla přijata, vytvořili portál kde se sdílí soudní rozhodnutí a právo ohledně kyberkriminality pro informační důvody, dále dělají vzdělávací moduly na organizovaný zločin včetně kyberkriminality (společné vzdělávání vyšetřovatelů nebo rovojových zemí)
    \item Rada Evropy --tvrdší charakter než snahy UNODC, skupina odborníků která pracuje na druhém dodatkovém protokolu k Úmluvě o kyberkriminalitě, který by měl řešit problém s omezenou možností spolupráce (nové procesní nástroje pro efektivní předávání dat, videokonference, mechanismus předávání dat, apod)
    \item EU -- vytváří certifikační mechanismus nástrojů pro vyšetřování kyberkriminality a nástrojů pro předávání el. důkazů, neexistuje jednoznačný postup pro předání el. důkazu (někdy se převáží na disku autem, někdy elektronicky, s tím ale mohou mít soudy problém → není to nutně bezpečné)
\end{itemize}
