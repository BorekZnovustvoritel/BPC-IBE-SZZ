\section[Software jako předmět právní ochrany\,--\,srovnání ochrany autorskoprávní a patentové ochrany]{Software jako předmět právní ochrany\,--\,srovnání och-rany autorskoprávní a patentové ochrany}









\textbf{Software} je souhrnný název pro všechny počítačové programy používané v počítači, které provádějí nějakou činost.











\textbf{Počítačový program} neví v českém právu přesně definován, ale popisuje se jako \uv{program v jakékoliv formě, včetně těch, které jsou součástí technického vybavení (HW)}. Jak Evropský patentový úřad tak SK jej vyjadřuje jako \uv{serii instrukcí, kterou lze spustit na počítači}.

Autor má práva vycházející z autorských práv, kde jsou tyto práva chráněna jako \textbf{literární dílo} bez ohledu na formu vyjádření.

Ochrana pokrývá vyjádření ve formě:
\begin{itemize}[noitemsep]
    \item strojový kód
    \item zdrojový kód
    \item jejich mezistupně
    \item přípravné koncepční materiály vzniklé při vývoji
    \begin{itemize}[noitemsep]
        \item model architektury softwaru
        \item funkční specifikace
        \item apod.
    \end{itemize}
\end{itemize}

Autorským právem nejsou chráněny myšlenky a principy na nichž je založen jakýkoliv prvek počítačového programu, včetně těch, které jsou podkladem jeho propojení z jiným programem. Takže není chráněna funkcionalita programu, ale pouze její objektivní vyjádření v podobě kódu.

Autorské právo rozlišuje \textbf{osobnostní a majetková práva}. Do osobnostních spadá možnost rozhodnou o zveřejnění díla, osobovat si autorství, rozhodnout o nedotknutelnosti díla. Autor se nemůže vzdát osobnostních práv, jelikož jsou nepřeveditelná. Do majetkových práv spadá rozhodnutí o užití díla (rozmnožení, rozšiřovaní, pronájem, půjčovaní aj.). Osobnostní jsou popsána v \href{https://www.zakonyprolidi.cz/cs/2000-121#p11}{Autorský zákon §11} a majetková  \href{https://www.zakonyprolidi.cz/cs/2000-121#p12}{Autorský zákon §12}, kdy jsou v dalších paragrafech popsány jednotlivé typy užití. 

Jak už bylo zmíněno \textbf{autorskoprávní ochrana} programů je neschopna chránit funkcionalitu daného programu. Dokáže chránit pouze objektivní vyjádření v~kódu, popřípadě jeho vizuální stránku, ale samostatná funkcionalita není chráněna. K přiměřené ochraně samostatné funkcionality je použita \textbf{patentová ochrana}. Patentově nelze chránit počítačové programy ale tzv. vynálezy uskutečňované počítačem.
\newline

\noindent Podmínky patentovatelnosti:
\begin{itemize}[noitemsep]
    \item vynález (ne program!) realizovaný počítačem
    \item novost -- vynález se považuje za~nový, není-li součástí stavu techniky
    \item výsledek vynálezecké činnosti
    \item průmyslová využitelnost -- může-li být vynález vyráběn nebo využíván ve všech odvětvích průmyslu
\end{itemize}
Obecně v~ČR je software chráněn pouze autorským zákonem, pokud nespadá pod \textbf{vynálezy realizované počítačem}, ty lze chránit pomocí patentů.  
Vynálezy uskutečňované počítačem jsou chráněny i v ČR. Dle zákona nelze patentovat \uv{\emph{plány, pravidla a způsob vykonávaní duševní činnosti, hraní her nebo vykonávání obchodní činnosti, jakož i programy počítačů}}.

Do vynálezu realizovaného počítačem lze zahrnout počítačový program myšlený jako produkt. Podmínkou je technický charakter příslušného vynálezu. Počítačový program je vynález realizovaný počítačem, jestliže je schopný vyvolat dodatečný technický účinek, když běží na počítači nebo je na něm nahrán. Nesmí se jednat o běžnou interakci mezi SW a HW. Nelze patentovat software, jako takový ale musí splňovat určité podmínky, jejich společným jmenovatelem je přítomnost dalšího technického prvku. 

Hlavní rozdíly mezi autorským právem a patentem.

\begin{center}
    \begin{tabular}{|l|l|}
    \hline
        Autorské právo & Patent \\\hline
        Získaní je automatické & Musí se o něj požádat\\\hline
        Zaniká 70 let po smrti autora & Vydává se na určitou dobu\\\hline
        Ochrana jako dílo literární & Ochraňuje myšlenky, metody, postupy\\\hline
        Neváže se s žádným poplatkem & Na každé podání se váže poplatek.\\\hline
        Zamezuje neoprávněnému šíření & \\\hline
    \end{tabular}
\end{center}









\newpage
\section{Autorství software\,--\,autor, spoluautor, kolektivní dílo, souborné dílo, odvozené dílo, otázka autorství umělé inteligence}
\textbf{Autorem} může být jakákoliv fyzická osoba, která dílo vytvořila. \href{https://www.zakonyprolidi.cz/cs/2000-121#p5}{Autorský zákon §5}

Zákonná domněnka autorství\,--\,Autorem díla je fyzická osoba, jejíž pravé jméno je obvyklým způsobem uvedeno na díle nebo je u díla uvedeno v rejstříku předmětů ochrany vedeném příslušným kolektivním správcem, není-li prokázán opak. \href{https://www.zakonyprolidi.cz/cs/2000-121#p6}{Autorský zákon §6}

Anonym a pseudonym\,--\,totožnost autora není možné zveřejnit bez jeho souhlasu. Jeho práva zastupuje osoba, která dílo zveřejnila. \href{https://www.zakonyprolidi.cz/cs/2000-121#p7}{Autorský zákon §7}

\textbf{Spoluautorem} je autor, který se podílel svou tvůrčí činností na vytvoření díla, společně s dalšími autory. Všem zúčastněným autorům připadá stejné právo a o nakládání s dílem musí být rozhodnuto jednomyslně. Bráni-li spoluautor bez vážného důvodu může jeho souhlas být doplněn soudně.

O spoluautorské dílo se jedná pokud jednotlivé části nejsou způsobilé samostatného užití, jedna~část potřebuje druhou.

Spoluautorem není osoba, která přispěla k vzniku díla pouze poskytnutím pomoci nebo rady technické, administrativní nebo odborné povahy. Dále poskytnutím dokumentačního nebo technického materiálu. Nakonec ten, kdo dal pouze podnět ke vzniku díla. \href{https://www.zakonyprolidi.cz/cs/2000-121#p8}{Autorský zákon §8}

\textbf{Kolektivní dílo} (\href{https://www.zakonyprolidi.cz/cs/2000-121#p59}{Autorský zákon §59}) je dílo, na~kterém se podílí více autorů a~je vytvořeno z podnětu a~pod vedením FO/PO a~uváděno na~veřejnost pod jejím jménem. Považují se často za~zaměstnanecká díla.

\textbf{Souborné dílo}\,--\,soubor nezávislých děl nebo prvků, který je způsobem výběru či uspořádáním obsahu jedinečným výsledkem tvůrčí činnosti autora (např. sborník, časopis, encyklopedie, \textbf{databáze} apod.). V souborném díle je chráněn způsob výběru nebo struktura (např. výběr hesel v encyklopedii či tematické uspořádání prvků). Autorem je fyzická osoba, která tvůrčím způsobem vybrala nebo uspořádala prvky souborného díla a tím nejsou dotčena práva autorů děl, které byli do souborného díla zařazeny. \href{https://www.zakonyprolidi.cz/cs/2000-121#p2}{Autorský zákon §2}

\textbf{Odvozené dílo}\,--\,je nové dílo, které vychází z již existujícího díla, které je chráněno autorským zákonem. Autorská práva původního vlastníka nejsou nijak dotčena, ale zároveň musí dát souhlas k vzniku odvozeného díla. Například při překladu knihy. U SW to znamená například oprava, rozšíření SW mimo autora, tak musí být udělena licence. 

\vspace{1cm}
\noindent\textbf{AI}

\noindent Momentálně dle autorského zákona~může být autorem pouze FO -- pokud SW vytvoří SW tak primární SW není autorem sekundárního SW. Pokud je při vývoji SW vysoká účast člověka, je sekundární SW brán jako výsledek lidské činnosti a~AI je použito jako pomocník při vývoji. v~tom případě je sekundární SW chráněn autorským právem. Při nízké účasti člověka~není rozhodnutí jednoznačné. Porovnávají se prvky objektivní vnímatelnosti, tvůrčí svobody, možnosti vyjádření apod. Obecně AI jejich výtvorů není definována. 

Rozlišujeme dva~druhy AI -- silné a~slabé. Slabá AI -- nefunguje jako blackbox, neboli dokážeme predikovat výstup a~víme, jak funguje na~pozadí. Silná AI -- funguje jako blackbox takže nevíme co se děje na~pozadí. U slabé AI je vlastníkem licence autor AI, zatímco při silné AI se žádný případ doposud neřešil, ani není stanovené nějaké právo takže se momentálně nedá jednoznačně říci.

\textbf{SW paradox} -- máme vývojáře, který vytvoří AI1 a~to licencuje uživateli 1. AI1 vytvoří AI2 a~tu bude využívat uživatel 2 a~v~té chvíli vzniká otázka, kdo ho má licencovat. Jestli vývojáři, uživatel 1 nebo AI1. Neví se přímo, na~koho se obrátit a~jestli nevyžadovat i licenci k AI1. Z tohoto plyne jakási nejistota~o platnosti licence. Autorské právo je postaveno na~objektivním pravidle. Kdo něco vytvořil tvůrčí činností, je autor a~může produkt licencovat. Jelikož si nejsme jistí od koho licencovat a~jestli se vůbec jedná o tvůrčí výsledek, může tato skutečnost paralyzovat jakoukoliv~licenční smlouvu a~její užívaní.

AI jako objekt práva~nemá právní osobnost a~způsobilost. Momentálně chybí přesvědčivý argument. 

Shrnutí: nejvíce relavantními jsou nároky tvůrců AI a~uživatelů -- něco jako spoluautorství. AI může být \uv{tvůrčí}, ale ne autor.










\newpage
\section{Osobnostní a majetková práva autora software -- obsah, doba trvání, vyčerpání práv}














\newpage
\section[Omezení rozsahu práv autora k software\,--\,dekompilace, reverzní inženýrství a další]{Omezení rozsahu práv autora k software\,--\,dekompi-lace, reverzní inženýrství a další}

Dekompilace programu, není povolena~v~každém případě. Lze provádět pouze za~účelem interoperability (je schopnost různých systémů vzájemně spolupracovat, poskytovat si služby, dosáhnout vzájemné součinnosti) a~je třeba~provádět jen nezbytně nutné úkony. Dekompilovat lze pouze až po vyčerpání všech možností jako je obrácení se na~autora. Platí pouze za~speciálních podmínek. Při dekompilaci lze pouze provádět rozmnožovaní kódu a~překlad formy kódu ve smyslu čl. 4 SW směrnice -- stále nebo dočasné rozmnožovaní, překlady, zpracovaní, a~jiné úpravy programu. Taktéž nelze dekompilovat pokud program, se kterým chceme dosáhnout interoperability neexistuje alespoň ve formě návrhu. Dále může dekompilovat pouze oprávněná osoba, nelegální držitelé licence dekompilaci nemohou provádět. Všechny potřebné informace mohou být použity pouze k dosažení interoperability. K tomu si musíme definovat, kdo je to oprávněný nabyvatel a uživatel.

Reverzní inženýrství lze provádět buď na základě dekompilace nebo blackbox a whitebox testovaní. 

\textbf{Oprávněný nabyvatel} -- může být ten, kdo si program zakoupil, pronajal, získal licenci přímo od držitele práv, a~i od původního oprávněného nabyvatele.

\textbf{Oprávněný uživatel} -- problematické, oprávněný uživatel může být na základě smlouvy, nebo zákona~\href{https://www.zakonyprolidi.cz/cs/2000-121#p66-6}{§66} \uv{\emph{Oprávněným uživatelem je oprávněný nabyvatel rozmnoženiny, který k ní má vlastnické nebo jiné právo za~účelem využití}}. Tato definice je tautologická nebolí říká že oprávněným uživatelem je oprávněný uživatel. Řešil to nejvyšší soud i~soudní dvůr EU, který stanovil, že se může jednat i o jinou osobu než o tu se kterou byla~smlouva~uzavřena~(lze přeprodat licenci), ale původní uživatel musí odstranit SW ze svého PC a~nepoužívat jej. Nejvyšší soud se zabýval poslaným CD s licenčním klíčem na produkty Microsoftu, v kterém vzal v potaz i rozhodnutí soudního dvora EU.

\textbf{Třístupňový test} -- hledání výjimky, platné jsou pouze výjimky stanovené v~zákoně, aplikace vyjímek je dovolena~jen, pokud to není v~rozporu s užitím díla~a~nejsou tím nepřiměřeně dotčeny oprávněné zájmy autora. I při dovoleném reverzním inženýrství musíme projít tímto testem.

\textbf{Blackbox analýza} -- mohu zkoumat nebo studovat funkčnost programu za~účelem zjištění myšlenek a~principů na~nichž je založen jakýkoliv~prvek programu a činím-li tak při zavedení, uložení do paměti počítače nebo při jeho zobrazení, provozu či přenosu, ke kterému je uživatel oprávněn. Není zde omezení jak nakládat se získanými informacemi. Charakteristika je, že není přístup ke zdrojovému kódu.  (\href{https://www.zakonyprolidi.cz/cs/2000-121#p66}{Autorský zákon §66} písmeno d)

\textbf{Whitebox analýza} -- je charakteristický tím že je znám zdrojový kód a můžeme si tedy představit co očekávat na výstupu při určitém vstupu na rozdíl od blackboxu. Zde je to podobné jako u dekompilace takže je povoleno pouze k dosažení interoperability a musí se projít třístupňovým testem.










\newpage
\section{Ochrana neliterárních složek software\,--\,ochrana funkcionality; ochrana datových, grafických uživatelských a aplikačních programových rozhraní)}

K přiměřené ochraně samostatné funkcionality je použita~\textbf{patentová ochrana}. Patentově nelze chránit počítačové programy ale tzv.\ vynálezy uskutečňované počítačem. Není tedy použita autorskoprávní ochrana.

Je uvedeno, že myšlenky a~principy, na~nichž je založen prvek programu, nejsou chráněny autorským zákonem. SW směrnice -- Myšlenky a~zásady na~kterých je založen kterýkoliv~prvek programu, nebo jeho rozhraní, nejsou chráněny. 

\textbf{Datová} -- jedná se o rozhraní, která slouží k ukládaní a~přenosu dat v~určitém formátu. Dle SW směrnice tyto rozhraní nebudou chráněna. Pokud by bylo bráno jako normální dílo, lze uvažovat o standardní ochraně.

\textbf{Uživatelská (GUI)} -- Při převzetí GUI se nejedná o rozmnoženinu počítačového programu. \textrightarrow Nejedná se o zásah do vyhrazeného práva. Pokud bychom ale naplnili při tvorbě GUI podmínku jedinečnosti, může být chráněno podle obecné autorskoprávní ochrany. Vzniká problém u právních systémů neoperujících s pojmem originálnost, např. UK. Tam GUI chráněna~nejsou.

\textbf{Aplikační (API)} -- Není specifikováno, zda~se jedná o myšlenku nebo vyjádření. SDEU neřešil ale lze předpokládat, že by to bylo podobné jako u GUI. Pokud by se jednalo o vyjádření tak je chráněno. Momentálně Google vs Oracle, o kterém by měl soud rozhodnout v~2021. Google pro Android využil stejné API jako je v~Javě. v~prvním řešení bylo rozhodnuto že API není chráněno. Poté odvolací soud rozhodl že API je chráněno, že byla~splněna~podmínka~originality. Momentálně je případ u nejvyššího soudu USA. \textbf{UPDATE} V dubnu 2021 bylo rozhodnuto ve prospěch Google, kde použití API v tomto případě bylo označeno na \uv{fair use}, kdy se začalo s možností, že API může podléhat autorskoprávní ochraně. Ve zkratce soud nerozhodl jestli je API pod autorskoprávní ochranou ale pouze o tom, že ze strany Google nešlo o porušení.  




\newpage
\section{Software jako zaměstnanecké dílo, školní dílo, kolektivní dílo a dílo na objednávku}











\newpage
\section{Právní ochrana databází\,--\,pojem, způsoby ochrany, obsah a rozdíly}











\newpage
\section[Smlouvy a software\,--\,licenční smlouva, smlouva o~dílo a SLA\,--\,pojem, strany, obsah, forma a proces uzavírání]{Smlouvy a software\,--\,licenční smlouva, smlouva o~dí-lo a SLA\,--\,pojem, strany, obsah, forma a proces uzavírání}

\begin{Large}
\textbf{Licenční smlouva}
\end{Large} 

Licenční smlouva~je smlouva, na~jejíž základě poskytovatel poskytuje oprávnění k užítí všech nebo jednotlivých způsobů užití. Nabyvatel se zavazuje poskytnout odměnu, není-li sjednáno jinak. 

Licenční smlouva~nemusí být v~písemné formě. Lze ji uzavřít například ústně. Musí být uzavřena~písemně pouze v~případech, kdy je poskytována~jako výhradní. V~případě výhradní licence autor nesmí poskytnout licenci třetí osobě a~je obvykle povinen nepoužívat SW, ke kterému výhradní licenci udělil. V~případě nevýhradní licence může autor používat SW a~k obsahu licence poskytnout licence dalším osobám. 
\newline

\noindent\textbf{Elektronické uzavíraní:}
\begin{itemize}[noitemsep]
    \item Click-wrap
    \begin{itemize}[noitemsep]
        \item Potvrzení před prvním užitím
    \end{itemize}
    \item Shrink-wrap
    \begin{itemize}[noitemsep]
        \item Rozbalení krabicového SW
    \end{itemize}
    \item Browse-wrap
    \begin{itemize}[noitemsep]
        \item Souhlas před stažením SW
    \end{itemize}
\end{itemize}

\noindent\textbf{Obsah licence:}
\begin{itemize}[noitemsep]
    \item Základní
    \begin{itemize}[noitemsep]
        \item Smluvní strany
        \item Specifikace autorského díla
        \begin{itemize}[noitemsep]
            \item Předmět
            \item Není nutno popisovat funkcionalitu
        \end{itemize}
        \item Právo a~způsob užití
        \item Rozsah licence
        \item Odměna~za~poskytnutí licence
        \item Přiměřená dodatečná odměna
        \item Délka~trvání licence
    \end{itemize}
    \item Ostatní
    \begin{itemize}[noitemsep]
        \item Právo podlicencování, či přeprodání
        \item Odpovědnost za~škodu a~právní vady SW
        \item Oprávnění k rozmnožovaní nebo úpravě SW
        \item Nárok na~upgrade SW
        \item Způsob zániku licence a~postupu po zániku
        \item Automatické prodlužovaní licence
    \end{itemize}
\end{itemize}

Hlavním účelem je ochrana~díla~a~specifikace, jak s ním lze nakládat. Například jestli lze upravovat nebo předělávat SW, rozmnožovat ho a~popřípadě upravenou verzi licencovat. 
\clearpage


\begin{Large}
\textbf{SLA Service Level Agreement}
\end{Large}


SLA~(= Service Level Agreement) je tzv. inominátní smlouva, která upravuje úroveň poskytovaní určité služby. Předmětem můžou být služby jako podpora, údržba~a~podobné spojené s dodávkou SW, služby v~oblasti cloud computingu nebo služby v~oblasti telekomunikací. Často není uzavíraná samostatně, ale je spíše doplňující smlouva.
\newline

\noindent\textbf{Typické prvky:}
\begin{itemize}[noitemsep]
    \item Vymezení samostatné služby, tedy její definice
    \begin{itemize}[noitemsep]
        \item Podpora~SW
        \item Odstraňovaní vad
        \item PC program poskytovaný jako služba
    \end{itemize}
    \item Parametry služby a~způsob vyhodnocení -- důležitá preciznost jejich vymezení.
    \begin{itemize}[noitemsep]
        \item Z pohledu parametrů je důležité přesně vymezit, kdy se využijí (např.\ výjimka z dostupnosti pro plánované odstávky).
        \begin{itemize}[noitemsep]
            \item Dostupnost
            \item Reakční doba
            \item Doba~do odstranění závad
        \end{itemize}
        \item  Z pohledu vyhodnocení je důležité jak bude provedeno vyhodnocení kvality služby
        \begin{itemize}[noitemsep]
            \item Období
            \begin{itemize}[noitemsep]
                \item Rok
                \item Měsíc
                \item Týden
                \item Apod.
            \end{itemize}
            \item Jaký mechanismus
            \begin{itemize}[noitemsep]
                \item Jak bude měřena~dostupnost služby
            \end{itemize}
        \end{itemize}
    \end{itemize}
    \item Kreditace -- klíčový prvek, forma~sankce za~nedodržení úrovně služby
    \begin{itemize}[noitemsep]
        \item Podoba
        \begin{itemize}[noitemsep]
            \item  Sleva~z ceny
            \item Smluvní pokuta~-- může překročit smluvní částku za~službu na~rozdíl od slevy
            \item Případně délka budoucího období poskytnutí služby zdarma~tzv~free service days
        \end{itemize}
    \end{itemize}
\end{itemize}

\begin{Large}
\textbf{Smlouva o dílo}
\end{Large}

Je to smlouva, na jejímž základě vzniká závazek, jehož předmětem je zhotovení, údržba, oprava nebo úprava věci či činnosti. Většinou se sepisuje pokud je činnost financovaná zákazníkem a~produkt (v~našem případě SW) je vytvářen dle požadavků zákazníka. Často se pojí se SW na zakázku.

Klíčovým ujednáním je specifikace předmětu plnění neboli SW\@. Oproti specifikaci standardního SW musí být míra specifikace o výrazně rozsáhlejší a~detailnější. Přesná specifikace musí proběhnout, aby se předešlo sporům. Proto by specifikace neměla pokrývat pouze funkcionalitu ale také parametry, které nejsou přímo spojené s funkcionalitou ale mohou ji zásadně ovlivnit, jako například HW nároky.

\vspace{1cm}
\textbf{Obsah smlouvy o dílo:}
\begin{itemize}[noitemsep]
    \item Zhotovitel a~zadavatel/objednatel
    \item Předmět smlouvy -- co zadavatel chce
    \item Cena
    \item Termín zhotovení -- do kdy a~co bude předáno
    \item Detailnější specifikace předání a~převzetí díla
    \item Odpovědnost za~vady
    \item Závěrečná ustanovení
\end{itemize}

Účelem uzavření licence by měla být nejen schopnost ho používat ale i různě s ním manipulovat. Například nemožnost přeprodávat nebo poskytovat SW třetí straně. Zde se aplikuje občanský zákoník, kde má zákazník možnost používat licenci pouze za účelem sjednaného ve smlouvě. Proto se často sjednává širší oprávnění zákazníka/široká licence. 
\vspace{1cm}








\newpage
\section{Veřejné licence a software, free a open source software}

Veřejná licence je specifickým způsobem sjednaná licenční smlouva. SW licencovaný pod veřejnou licencí je vetšinou poskytován bez úplaty, tímto způsobem se lze zbavit odpovědnosti za~chyby v~programu, které nezpůsobují právní vady. Všechny typy nejčastěji obsahují alespoň podmínku uvedení autora. 
\vspace{0.2cm}

Podstatou veřejné licence je zveřejnění díla~s licenčními podmínkami, odkazem na tuto konkrétní licenci, kde nabyvatel licence není v~přímém kontaktu s poskytovatelem. Využívá se hlavně v~situaci, kdy licenci chceme směřovat na neurčitý počet osob. \uv{\emph{Veřejné licence jsou veřejné návrhy k uzavření licenčních smluv, jejichž obsah je standardizován a~vymezen odkazem na~veřejně známé a~dostupné licenční podmínky a~určen neurčitému počtu osob}}.
\vspace{0.05cm}

Nejčastěji se veřejných licencí využívá ve FOSS (Free and opensource software). V~tomto případě free neznamená zdarma ale svobody jeho užívaní a opensource znamená, že je dostupný zdrojový kód. Poté existuje také proprietární SW, ten nejčastěji nezveřejňuje zdrojový kód a využívá jiné typy licencí než veřejné. Typy licencí mohou být silně copyleftové, slabě copyleftové a~necopyleftové.

\textbf{Silně copyleftové} nesou omezené při zpracovaní a~šíření SW. Požadují, aby původní nebo nový program, který obsahuje původní, byl šířen pod původními licenčními podmínkami a~současně garantuje tvůrci přístup k novému zdrojovému kódu. Zástupci jsou GNU GPL v2 a~v3.

\textbf{Slabě copyleftové} vyžadují šíření odvozených programů pod stejnými licenčními podmínkami a~zpřístupnění jejich zdrojových kódů. Umožňují vytváření programů, které jsou propojené a~šířené společně s původním programem aniž by měnily či používaly jeho zdrojový kód. Tyto programy lze šířit pod libovolnou licencí. Nejčastěji to jsou standardní knihovny. Nemusí se vydat zdrojové kódy vlastního kódu ale pouze se musí uvést a~zpřístupnit původní část programu pod původní licencí. Při vzniku pouze odvozeného díla by musela být použita stejná licence. Zástupci MPL (Mozila~Public License) v~1.1 a~ LGPL (Lesser General Public License) v2.1.

\textbf{Necopyleftové} licence neobsahují žádnou nebo velmi omezenou copyleftovou doložku. Ukládají pouze minimální omezení k dalšímu šíření. Proto lze použít i při vývoji SW s neveřejným zdrojovým kódem aniž by bylo porušeno původních podmínek. Zástupci Apache 2.0, BSD a~MIT\@.
\vspace{0.2cm}

Licence lze měnit směrem od nejslabší po nejsilnější ale ne naopak. Další často používanou licencí je Creative Commons. CC je licence, která má více úrovní omezení a nakládaní s dílem. Klade důraz na to, aby byli čitelná i bez podrobné studovaní licence. Tohoto dosahuje pomocí obrázku a zkratky uvedené veřejně. Nejčastěji používané jsou CC BY, která slouží aby byl uveden původní tvůrce, CC BY-SA sloužící k uvedení tvůrce a aby byla zachována licence. Všechny typy lze nalézt na jejich stránkách \href{https://creativecommons.org/licenses/}{zde}.
\vspace{0.2cm}

Pokud bychom měli SW v kterém je použito více licencí, tak musíme rozebrat jednotlivé licence a zjistit do jaké skupiny spadají a co z nich vyplývá. Podle toho nakonec můžeme určit pod jakou licencí lze vydat SW.








\newpage
\section{Autorskoprávní a trestněprávní prostředky ochrany software\,--\,účel, nároky, tresty a realizace}

\newpage